% Options for packages loaded elsewhere
\PassOptionsToPackage{unicode}{hyperref}
\PassOptionsToPackage{hyphens}{url}
\PassOptionsToPackage{dvipsnames,svgnames,x11names}{xcolor}
%
\documentclass[
  12pt,
  a4paper,
  DIV=11,
  numbers=noendperiod]{scrartcl}

\usepackage{amsmath,amssymb}
\usepackage{iftex}
\ifPDFTeX
  \usepackage[T1]{fontenc}
  \usepackage[utf8]{inputenc}
  \usepackage{textcomp} % provide euro and other symbols
\else % if luatex or xetex
  \usepackage{unicode-math}
  \defaultfontfeatures{Scale=MatchLowercase}
  \defaultfontfeatures[\rmfamily]{Ligatures=TeX,Scale=1}
\fi
\usepackage{lmodern}
\ifPDFTeX\else  
    % xetex/luatex font selection
\fi
% Use upquote if available, for straight quotes in verbatim environments
\IfFileExists{upquote.sty}{\usepackage{upquote}}{}
\IfFileExists{microtype.sty}{% use microtype if available
  \usepackage[]{microtype}
  \UseMicrotypeSet[protrusion]{basicmath} % disable protrusion for tt fonts
}{}
\makeatletter
\@ifundefined{KOMAClassName}{% if non-KOMA class
  \IfFileExists{parskip.sty}{%
    \usepackage{parskip}
  }{% else
    \setlength{\parindent}{0pt}
    \setlength{\parskip}{6pt plus 2pt minus 1pt}}
}{% if KOMA class
  \KOMAoptions{parskip=half}}
\makeatother
\usepackage{xcolor}
\usepackage[top=20mm,left=20mm,heightrounded]{geometry}
\setlength{\emergencystretch}{3em} % prevent overfull lines
\setcounter{secnumdepth}{-\maxdimen} % remove section numbering
% Make \paragraph and \subparagraph free-standing
\ifx\paragraph\undefined\else
  \let\oldparagraph\paragraph
  \renewcommand{\paragraph}[1]{\oldparagraph{#1}\mbox{}}
\fi
\ifx\subparagraph\undefined\else
  \let\oldsubparagraph\subparagraph
  \renewcommand{\subparagraph}[1]{\oldsubparagraph{#1}\mbox{}}
\fi


\providecommand{\tightlist}{%
  \setlength{\itemsep}{0pt}\setlength{\parskip}{0pt}}\usepackage{longtable,booktabs,array}
\usepackage{calc} % for calculating minipage widths
% Correct order of tables after \paragraph or \subparagraph
\usepackage{etoolbox}
\makeatletter
\patchcmd\longtable{\par}{\if@noskipsec\mbox{}\fi\par}{}{}
\makeatother
% Allow footnotes in longtable head/foot
\IfFileExists{footnotehyper.sty}{\usepackage{footnotehyper}}{\usepackage{footnote}}
\makesavenoteenv{longtable}
\usepackage{graphicx}
\makeatletter
\def\maxwidth{\ifdim\Gin@nat@width>\linewidth\linewidth\else\Gin@nat@width\fi}
\def\maxheight{\ifdim\Gin@nat@height>\textheight\textheight\else\Gin@nat@height\fi}
\makeatother
% Scale images if necessary, so that they will not overflow the page
% margins by default, and it is still possible to overwrite the defaults
% using explicit options in \includegraphics[width, height, ...]{}
\setkeys{Gin}{width=\maxwidth,height=\maxheight,keepaspectratio}
% Set default figure placement to htbp
\makeatletter
\def\fps@figure{htbp}
\makeatother
% definitions for citeproc citations
\NewDocumentCommand\citeproctext{}{}
\NewDocumentCommand\citeproc{mm}{%
  \begingroup\def\citeproctext{#2}\cite{#1}\endgroup}
\makeatletter
 % allow citations to break across lines
 \let\@cite@ofmt\@firstofone
 % avoid brackets around text for \cite:
 \def\@biblabel#1{}
 \def\@cite#1#2{{#1\if@tempswa , #2\fi}}
\makeatother
\newlength{\cslhangindent}
\setlength{\cslhangindent}{1.5em}
\newlength{\csllabelwidth}
\setlength{\csllabelwidth}{3em}
\newenvironment{CSLReferences}[2] % #1 hanging-indent, #2 entry-spacing
 {\begin{list}{}{%
  \setlength{\itemindent}{0pt}
  \setlength{\leftmargin}{0pt}
  \setlength{\parsep}{0pt}
  % turn on hanging indent if param 1 is 1
  \ifodd #1
   \setlength{\leftmargin}{\cslhangindent}
   \setlength{\itemindent}{-1\cslhangindent}
  \fi
  % set entry spacing
  \setlength{\itemsep}{#2\baselineskip}}}
 {\end{list}}
\usepackage{calc}
\newcommand{\CSLBlock}[1]{\hfill\break\parbox[t]{\linewidth}{\strut\ignorespaces#1\strut}}
\newcommand{\CSLLeftMargin}[1]{\parbox[t]{\csllabelwidth}{\strut#1\strut}}
\newcommand{\CSLRightInline}[1]{\parbox[t]{\linewidth - \csllabelwidth}{\strut#1\strut}}
\newcommand{\CSLIndent}[1]{\hspace{\cslhangindent}#1}

\KOMAoption{captions}{tableheading}
\usepackage{wrapfig}
\usepackage{subcaption}
\usepackage{amsmath}
\usepackage{cancel}
\usepackage{hyperref}
\usepackage{tikz}
\usepackage{tabularx}
\usepackage{colortbl}
\usepackage{xcolor}
\renewcommand{\maketitle}{}
\definecolor{cornflowerblue}{RGB}{100,149,237}
\definecolor{darkgrey}{RGB}{220,220,220}
\usepackage{fancyhdr}
\pagestyle{fancy}
\fancyhf{}
\fancyhead[L]{\rightmark}
\fancyhead[R]{\thepage}
\fancyfoot[C]{\thepage}
\makeatletter
\@ifpackageloaded{caption}{}{\usepackage{caption}}
\AtBeginDocument{%
\ifdefined\contentsname
  \renewcommand*\contentsname{Table of contents}
\else
  \newcommand\contentsname{Table of contents}
\fi
\ifdefined\listfigurename
  \renewcommand*\listfigurename{Figurliste}
\else
  \newcommand\listfigurename{Figurliste}
\fi
\ifdefined\listtablename
  \renewcommand*\listtablename{Tabelliste}
\else
  \newcommand\listtablename{Tabelliste}
\fi
\ifdefined\figurename
  \renewcommand*\figurename{Figur}
\else
  \newcommand\figurename{Figur}
\fi
\ifdefined\tablename
  \renewcommand*\tablename{Tabell}
\else
  \newcommand\tablename{Tabell}
\fi
}
\@ifpackageloaded{float}{}{\usepackage{float}}
\floatstyle{ruled}
\@ifundefined{c@chapter}{\newfloat{codelisting}{h}{lop}}{\newfloat{codelisting}{h}{lop}[chapter]}
\floatname{codelisting}{Listing}
\newcommand*\listoflistings{\listof{codelisting}{List of Listings}}
\makeatother
\makeatletter
\makeatother
\makeatletter
\@ifpackageloaded{caption}{}{\usepackage{caption}}
\@ifpackageloaded{subcaption}{}{\usepackage{subcaption}}
\makeatother
\ifLuaTeX
  \usepackage{selnolig}  % disable illegal ligatures
\fi
\usepackage{bookmark}

\IfFileExists{xurl.sty}{\usepackage{xurl}}{} % add URL line breaks if available
\urlstyle{same} % disable monospaced font for URLs
\hypersetup{
  colorlinks=true,
  linkcolor={blue},
  filecolor={Maroon},
  citecolor={Blue},
  urlcolor={Blue},
  pdfcreator={LaTeX via pandoc}}

\author{}
\date{}

\begin{document}


\newgeometry{left=0cm, right=0cm, top=0cm, bottom=0cm}
\vspace*{0.5cm} 
\hspace*{1.5cm}\includegraphics[width=10cm]{dokumentobjekter/texstuff/UiT_Logo_Bok_Bla_RGB.png} 


\begin{flushleft}
    \vspace*{0.5cm}
    \hspace*{2.5cm}{\color{black}\fontsize{11}{13.2}\selectfont Handelshøgskolen ved UiT \\[0.2em]
    \hspace*{2.5cm}\color{black}\fontsize{8}{13.2}\selectfont Fakultet for biovitenskap, fiskeri og økonomi \\[0.2em]
    \hspace*{2.5cm}\large{\color{black}\textbf{Obligatorisk oppgave}}  \\[0.5em]
    \hspace*{2.5cm}\color{black}\fontsize{12}{14.4}\selectfont Eksperimentell metode \\[0.5em]
\hspace*{2.5cm}\color{black}\fontsize{11}{13.2}\selectfont Daniel Johannessen og Daniel Groth \\[0.5em]
    \hspace*{2.5cm}\color{black}\fontsize{11}{13.2}\selectfont Sok-2012, Vår 2024 \\[0.5em]
    \hspace*{2.0cm}
    \par}
\end{flushleft} 



\begin{tikzpicture}[remember picture, overlay]
    \node[anchor=south west, inner sep=0] at (current page.south west) {\includegraphics[width=\paperwidth]{dokumentobjekter/texstuff/forside_bilde.png}};
\end{tikzpicture}


\newgeometry{left=20mm, right=20mm, top=20mm, bottom=20mm}




\thispagestyle{plain}
\begin{center}
    \Large
    \textbf{Obligatorisk oppgave}
\end{center}


Denne oppgaven handler om å desinge et eget eksperiment. Spørmålene er ment for å strukturere arbeidsporssesen deres. Dere kan velge å lever oppgave i grupper fra 1 (alene) til 3. 



Send en epost før 15. Mars til
oivind.d.schoyen@uit.no om hvilken gruppe dere er del av.
Innleveringsfristen for oppgaven er 15. Mai. 2024. Klokken 15:30.











\newpage
\hypersetup{linkcolor=black}
\renewcommand{\contentsname}{Innholdsfortegnelse}
\renewcommand*{\figureautorefname}{Figur}
\renewcommand*{\tableautorefname}{Tabell}
\tableofcontents
\hypersetup{linkcolor=blue}
\newpage

\subsection{Eksperimentell metode}\label{eksperimentell-metode}

\subsubsection{1. A.}\label{a.}

\paragraph{Hvilke spørsmål om menneskelig adferd kan besvares ved hjelp
av eksperimentelle
metoder?}\label{hvilke-spuxf8rsmuxe5l-om-menneskelig-adferd-kan-besvares-ved-hjelp-av-eksperimentelle-metoder}

Kan ikke teste mot kjønn for eksempel siden du ikke kan kontrollere
hvilket kjønn noen er i. Medfødte egenskaper kan ikke kontrolleres. kan
ikke variere høyde.

Kan bare besvare ting der det er ting vi kan variere. kan ikke besvare
ting der spørsmålet gjelder noe vi ikke kan variere som medfødte
egenskaper.

\subsubsection{1. B Forklar hva en RCT er og hva det er godt
for?}\label{b-forklar-hva-en-rct-er-og-hva-det-er-godt-for}

Randomized Controlled Trial. Du har en kontrollgruppe og en
eksperimentell gruppe. Du randomiserer individer inn i gruppene. Det er
godt for å teste effekten av en behandling da du kan teste effekten ved
å se på forskjellen mellom gruppene. teste kausalitet

\subsection{Test om mennesker gjør rasjonelle
valg}\label{test-om-mennesker-gjuxf8r-rasjonelle-valg}

\subsubsection{2.1. Hvilken effekt har et kurs i sannsynlighet på valg
av konvolutter i et
eksperiment?}\label{hvilken-effekt-har-et-kurs-i-sannsynlighet-puxe5-valg-av-konvolutter-i-et-eksperiment}

Vi gir folk 100kr, og starter med å fortelle de om premissene om et
spill hvor de får et valg mellom 3 konvolutter der en av konvoluttene
inneholder 150kr og to av de inneholder ingenting. Før de låser inn sitt
svar så forteller vi at vi vet hvilken konvolutt pengene ligger i og
åpner en av de med ingenting i seg, De får så en mulighet til å bytte
valg av konvolutter. De blir så gitt et valg om å spille det spillet men
at det koster de 50 kroner av de 100 kroner de fikk.

Vi gir igjen folk 100kr men denne gangen så gir vi den ene gruppen et
kurs i sannsynlighet først og den andre gruppen får ikke kurset. Vi
ønsker å se om de som har fått kurset i sannsynlighet tar bedre valg enn
de som ikke har fått kurset.

\clearpage

\subsubsection{2.2. Bruk google scholar til å finne en eller to
publiserte vitenskapelig studie (teoretisk eller empirisk) som forklarer
hvorfor X kan ha en effekt på Y
.}\label{bruk-google-scholar-til-uxe5-finne-en-eller-to-publiserte-vitenskapelig-studie-teoretisk-eller-empirisk-som-forklarer-hvorfor-x-kan-ha-en-effekt-puxe5-y-.}

\href{https://www.tandfonline.com/doi/full/10.1080/1445979042000224377}{Impact
of warning and brief intervention messages on knowledge of gambling
risk, irrational beliefs and behaviour}

``In contrast to those who watched the video only, participants in the
two message conditions showed greater knowledge of the risks of
gambling. The limit‐setting strategy produced significant reductions in
gambling‐related irrational beliefs. Across conditions, participants did
not gamble differently. These results suggest that warning messages
might have informational value and that limit‐setting strategies hold
promise for producing cognitive change in gamblers. Under the present
analog procedure, such messages did not significantly affect gambling
behaviour.''

Studien til Steenbergh et al. (2004) viser at å informere mennesker på
forhånd om risikoen av gambling ikke ga noen signifikante effekter på
spilleavhengighet.

\href{https://www.tandfonline.com/doi/abs/10.1080/00207590600788047}{The
effect of knowledge of mathematics on gambling behaviours and erroneous
perceptions}

``The importance of knowledge of mathematics as a protective factor
against excessive gambling is questionable. The theoretical and
practical implications of these results are discussed with regard to the
prevention of excessive gambling.''

Studien til Pelletier \& Ladouceur (2007) viser at å kunne matematikken
eller forstå oddsen og bruke det som en beskyttende faktor mot
spilleavhengighet er tvilsom.

\href{https://www.sciencedirect.com/science/article/pii/0030507379900114}{The
role of statistical knowledge in gambling decisions: Moment vs risk
dimension approaches}

``This study suggests that the conflicting findings of previous studies
can be partly explained by controlling for the confounding effects of
statistical training. It is suggested that statistical knowledge (1)
reduces the cognitive complexity of assessing duplex bets, (2) improves
the quality of risk-assessment, and (3) increases the propensity that
subjects will follow a moment model. The findings were corroborated by
regression analyses, although note was made of the methodological
difficulties inherent in this approach.''

Studien til Schoemaker (1979) sier at å kunne statistikk eller ha blitt
kurset i det kan vise til redusert kognitive kompleksit ved vurdering av
duplex-spill og forbedrer kvaliteten av risiko vurdering.

\href{https://psycnet.apa.org/record/2006-03168-007}{Does learning about
the mathematics of gambling change gambling behavior?}

``The present research examined the influence of improved knowledge of
odds and mathematical expectation on the gambling behavior of university
students. A group of 198 students in an introductory statistics class
received instruction on probability theory using examples from gambling.
A comparison group of 134 students received generic instruction on
probability, and another group of 138 students in classes on unrelated
topics received no mathematical instruction. Students receiving the
intervention demonstrated superior ability to calculate gambling odds as
well as resistance to gambling fallacies 6 months after the
intervention. Unexpectedly, this improvement in knowledge and skill was
not associated with any decreases in actual gambling behavior. The
implication of this research is that enhanced mathematical knowledge on
its own may be insufficient to change gambling behavior.''

Studien til Williams \& Connolly (2006) sier at de studentene som lærte
om statistikk og sannsynlighet viste bedre resultater for å kalkulere
gambling odds og hadde større motstandsdyktighet for å ikke bli avhengig
ovenfor gruppen som ikke ble kurset. Men den viste også at selv om de
ble kurset så hadde det ikke noen effekt på nedgang i spilleatferd.

\clearpage

\subsection{3. Design ett eksperiment.}\label{design-ett-eksperiment.}

Der du måler effekten av X på Y ved hjelp av en kontrollgruppe der
individer tilfeldig trekkes inn i gruppe \(\overline T\) eller gruppe
\(T\) .

X: Sannsynligheten for å velge å bytte konvolutt i Monty Hall-problemet,
målt som andelen deltakere som velger å bytte etter å ha mottatt
informasjon om innholdet i en av de andre konvoluttene. Y: Deltakelse i
et kurs i sannsynlighet

Vi randomiserer deltakerne til enten kontrollgruppen eller
eksperimentgruppen. Kontrollgruppen mottar ikke noe kurs, mens
eksperimentgruppen deltar i et kurs i sannsynlighetsteori.

Intervensjon: Kurset i sannsynlighet går raskt over grunnleggende
sannsynlighetsregler, forventningsverdier, uavhengighet, og en grundig
gjennomgang av Monty Hall-problemet og Bayes' teorem.

Eksperimentell oppsett: Før kurset gis begge gruppene 100 kr hver. De
får deretter et tilbud om å bruke 50 kr for å delta i et Monty
Hall-lignende spill. I spillet velger de en av tre konvolutter. En av
konvoluttene inneholder 150 kr, mens de to andre er tomme.
Eksperimentlederen åpner deretter en av de to resterende konvoluttene
som er tom og gir deltakeren valget om å bytte til den gjenværende
konvolutten eller beholde det opprinnelige valget. Dette gjøres så igjen
for begge gruppene etter at eksperimentgruppen har deltatt i kurset. Vi
ser da på differansen i andelen som bytter konvolutt mellom de som har
tatt kurset og de som ikke har tatt kurset både før og etter kurset.

\clearpage

\subsection{Effekt og analyse}\label{effekt-og-analyse}

\subsubsection{4. Forklar enklest mulig hvordan oppsetet ditt viser
effekten av X på Y
.}\label{forklar-enklest-mulig-hvordan-oppsetet-ditt-viser-effekten-av-x-puxe5-y-.}

Et mål \(M_t\) av effekten X på Y ved å studere
\(M_{\overline T} − M_{\underline T}\) .

Når vi nå ser hvilke valg de 2 gruppene har tatt, så kan vi se om det
har vært en signifikant endring i om de velger å kaste terningen eller
ikke.

\subsubsection{Bevis på best valg}\label{bevis-puxe5-best-valg}

Siden vi vet at konvolutten som åpnes er tom så setter vi opp Monty Hall
problemet i med Bayes. Da kan vi se hvordan denne informasjonen påvirker
eller ``oppdaterer'' sannsynligheten for at premien befinner seg i en
bestemt konvolutt.

For å vise dette matematisk så definerer vi hendelsene:

\begin{itemize}
\item
  \(G_i\): Hendelsen at gevinsten (150 kr) er i konvolutt \(i\) (hvor
  \(i\) kan være 1, 2, eller 3).
\item
  \(D_j\): Hendelsen at deltakeren velger konvolutt \(j\) (hvor \(j\)
  også kan være 1, 2, eller 3).
\item
  \(V_k\): Hendelsen at verten åpner konvolutt \(k\) (hvor \(k\) kan
  være 1, 2, eller 3) og viser at den er tom.
\end{itemize}

\subsubsection{1. Uavhengighet}\label{uavhengighet}

Vi antar at deltakeren velger en konvolutt tilfeldig, og at gevinsten er
plassert tilfeldig i en av de tre konvoluttene. Dette betyr at
hendelsene \(G_i\) og \(D_j\) er uavhengige for alle \(i\) og \(j\) .

Dette betyr at: \(P(G_i \mid D_j) = P(G_i)\) for alle \(i\) og \(j\).

\subsubsection{2. Initial
Sannsynligheter}\label{initial-sannsynligheter}

Før noe valg er gjort, er sannsynligheten for at gevinsten er i en
hvilken som helst konvolutt lik:
\[ P(G_i) = P(D_j) = P(V_k) = \frac{1}{3} \] for \(i, j, k = 1, 2, 3\).

\subsubsection{3. Avhengighet}\label{avhengighet}

\paragraph{Første valg blir gjort.}\label{fuxf8rste-valg-blir-gjort.}

Når vi åpner den andre konvolutten så er den avhengig av hvilken
konvolutt deltakeren har valgt og hvor gevinsten befinner seg. Vi åpner
alltid åpner en tom konvolutt og aldri den som deltakeren har valgt
eller den som inneholder gevinsten.

Dette betyr at: \[ P(V_k \mid G_i, D_j) \] er 0 hvis \(k = i\) eller
\(k = j\) (med mindre \(i = j\)), og ellers 1 delt på antall gjenværende
konvolutter.

\subsubsection{4. Bayes Teorem}\label{bayes-teorem}

Bayes teorem forteller oss hvordan vi kan oppdatere sannsynligheten for
at gevinsten er i en bestemt konvolutt gitt at verten åpner en annen
konvolutt: \[ 
P(G_i \mid V_k, D_j) = \frac{P(V_k \mid G_i, D_j) \cdot P(G_i)}{P(V_k \mid D_j)} 
\]

For å beregne \(P(V_k \mid D_j)\) under brøkstreken er sannsynligheten
for at verten åpner konvolutt \(k\), gitt at deltakeren har valgt
konvolutt \(j\).

Dette krever marginalisering over alle mulige steder gevinsten kan være:
\[ P(V_k \mid D_j) = \sum_{i=1}^3 P(V_k \mid G_i, D_j) \cdot P(G_i) \]

\subsubsection{Eksempel}\label{eksempel}

Anta at deltakeren velger konvolutt 1 (\(D_1\)) og verten åpner
konvolutt 3 ( \(V_3\) ), som er tom. Vi vil beregne
\(P(G_2 \mid V_3, D_1)\).

\begin{itemize}
\tightlist
\item
  \(P(V_3 \mid G_1, D_1) = 1/2\) fordi verten må velge mellom konvolutt
  2 og 3 når gevinsten er i konvolutt 1 og deltakeren har valgt
  konvolutt 1.
\item
  \(P(V_3 \mid G_2, D_1) = 1\) fordi verten bare kan åpne konvolutt 3
  når gevinsten er i konvolutt 2 og deltakeren har valgt konvolutt 1.
\item
  \(P(V_3 \mid G_3, D_1) = 0\) fordi verten aldri vil åpne konvolutten
  med gevinsten.
\end{itemize}

Så:

\[ 
P(V_3 \mid D_1) = P(V_3 \mid G_1, D_1) \cdot P(G_1) + P(V_3 \mid G_2, D_1) \cdot P(G_2) + P(V_3 \mid G_3, D_1) \cdot P(G_3) 
\]

\[ P(V_3 \mid D_1) = \frac{1}{2} \cdot \frac{1}{3} + 1
\cdot \frac{1}{3} + 0 \cdot \frac{1}{3} = \frac{1}{2} \]

Deretter:

\[ P(G_2 \mid V_3, D_1) = \frac{P(V_3 \mid G_2, D_1) \cdot P(G_2)}{P(V_3 \mid D_1)} = \frac{1 \cdot \frac{1}{3}}{\frac{1}{2}} = \frac{2}{3} \]

Dette viser at det er en \(\frac{2}{3}\) sannsynlighet for at gevinsten
er i den konvolutten deltakeren ikke valgte, som er konvolutt 2 i dette
tilfellet.

\clearpage

\subsubsection{5. Forklar hvordan du skal analysere dataene fra
eksperimentet ditt for å vise effekten av X på Y
.}\label{forklar-hvordan-du-skal-analysere-dataene-fra-eksperimentet-ditt-for-uxe5-vise-effekten-av-x-puxe5-y-.}

\paragraph{Forklar de grunnlegende premissene for eventuelle statistiske
metoder du
anvender.}\label{forklar-de-grunnlegende-premissene-for-eventuelle-statistiske-metoder-du-anvender.}

Vi kan bruke en z-test for å se om det er en signifikant forskjell i
valgene til de som har fått kurset i sannsynlighet, og de som ikke har
fått kurset gitt at vi får nok deltagere.

De grunnleggende premissene er da at vi får minimum 30 stykk i hver
gruppe og at de er uavhengige av hverandre.

Det kan også kanskje brukes en T test men vi er usikker siden dataen i
gruppene er binær.

\subsection{6}\label{section}

\subsubsection{6.1 Skriv en enkel pre-analyse plan for eksperimentet
ditt.}\label{skriv-en-enkel-pre-analyse-plan-for-eksperimentet-ditt.}

Preanalysis plan:

Vi gir folk 100kr, og starter med å fortelle de om premissene om et
spill hvor de får et valg mellom 3 konvolutter der en av konvoluttene
inneholder 150kr og to av de inneholder ingenting. Før de låser inn sitt
svar så forteller vi at vi vet hvilken konvolutt pengene ligger i og
åpner en av de med ingenting i seg, De får så en mulighet til å bytte
valg av konvolutter. De blir så gitt et valg om å spille det spillet men
at det koster de 50kr av de 100kr de fikk.

Vi gir igjen folk 100kr men denne gangen så gir vi den ene gruppen et
kurs i sannsynlighet først og den andre gruppen får ikke kurset. Vi
ønsker å se om de som har fått kurset i sannsynlighet tar bedre valg enn
de som ikke har fått kurset.

Vi vil så bruke en t-test for å se om det er en signifikant forskjell i
valgene til de som har fått kurset i sannsynlighet og de som ikke har
fått kurset.

\subsubsection{6.2 Forklar hvorfor det er viktig å publisere en
pre-analysis plan før du analyserer data. Hvordan hjelper pre-analysis
plan å tolke data for
utenforstående?}\label{forklar-hvorfor-det-er-viktig-uxe5-publisere-en-pre-analysis-plan-fuxf8r-du-analyserer-data.-hvordan-hjelper-pre-analysis-plan-uxe5-tolke-data-for-utenforstuxe5ende}

Det er viktig å publisere en pre-analyse plan før du analyserer data
fordi det er lett å manipulere dataene for å få det resultatet du
ønsker. Ved å publisere en pre-analyse plan så kan du vise at du ikke
har manipulert dataene og at du har fulgt planen du har lagt. Dette
hjelper utenforstående å tolke dataene fordi de kan se at du har fulgt
planen og at du ikke har manipulert dataene for å få det resultatet du
ønsker.

\clearpage

\section{Referanser}\label{referanser}

\phantomsection\label{refs}
\begin{CSLReferences}{1}{0}
\bibitem[\citeproctext]{ref-noauthor_pre-analysis_nodate}
A pre-analysis plan checklist. (n.d.). In \emph{World Bank Blogs}.
Retrieved May 11, 2024, from
\url{https://blogs.worldbank.org/en/impactevaluations/a-pre-analysis-plan-checklist}

\bibitem[\citeproctext]{ref-newall_statistical_2023}
Newall, P. W. S., Walasek, L., Hassanniakalager, A., Russell, A. M. T.,
Ludvig, E. A. \& Browne, M. (2023). Statistical risk warnings in
gambling. \emph{Behavioural Public Policy}, \emph{7}(2), 219--239.
\url{https://doi.org/10.1017/bpp.2020.59}

\bibitem[\citeproctext]{ref-pelletier_effect_2007}
Pelletier, M. \& Ladouceur, R. (2007). The effect of knowledge of
mathematics on gambling behaviours and erroneous perceptions.
\emph{International Journal of Psychology}, \emph{42}(2), 134--140.
\url{https://doi.org/10.1080/00207590600788047}

\bibitem[\citeproctext]{ref-schoemaker_role_1979}
Schoemaker, P. J. H. (1979). The role of statistical knowledge in
gambling decisions: {Moment} vs risk dimension approaches.
\emph{Organizational Behavior and Human Performance}, \emph{24}(1),
1--17. \url{https://doi.org/10.1016/0030-5073(79)90011-4}

\bibitem[\citeproctext]{ref-steenbergh_impact_2004}
Steenbergh, T. A., Whelan, J. P., Meyers, A. W., May, R. K. \& Floyd, K.
(2004). Impact of warning and brief intervention messages on knowledge
of gambling risk, irrational beliefs and behaviour. \emph{International
Gambling Studies}, \emph{4}(1), 3--16.
\url{https://doi.org/10.1080/1445979042000224377}

\bibitem[\citeproctext]{ref-williams_does_2006}
Williams, R. J. \& Connolly, D. (2006). Does learning about the
mathematics of gambling change gambling behavior? \emph{Psychology of
Addictive Behaviors}, \emph{20}(1), 62--68.
\url{https://doi.org/10.1037/0893-164X.20.1.62}

\end{CSLReferences}

\clearpage

\appendix

\section {Appendix Generell KI bruk}

I løpet av koden så kan det ses mange \# kommentarer der det er skrevet
for eks ``\#fillbetween q1 and q2''. Når vi skriver kode i Visual Studio
Code så finnes det en plugin som heter Github Copilot. Når vi skriver
slike kommentarer så kan den foresøke å fullføre kodelinjene mens vi
skriver de. Noen ganger klarer den det, men andre ikke. Det er vanskelig
å dokumentere hvert bruk der den er brukt siden det ``går veldig fort''
men siden det ikke er fått på plass en slik dokumentasjon så kan all
python kode der det er brukt kommentarer antas som at det er brukt
Github Copilot. Nærmere info om dette KI verktøyet kan ses på
\url{https://github.com/features/copilot}



\end{document}
