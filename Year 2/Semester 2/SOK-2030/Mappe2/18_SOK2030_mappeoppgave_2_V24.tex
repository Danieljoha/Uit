% Options for packages loaded elsewhere
\PassOptionsToPackage{unicode}{hyperref}
\PassOptionsToPackage{hyphens}{url}
\PassOptionsToPackage{dvipsnames,svgnames,x11names}{xcolor}
%
\documentclass[
  12pt,
  a4paper,
  DIV=11,
  numbers=noendperiod]{scrartcl}

\usepackage{amsmath,amssymb}
\usepackage{iftex}
\ifPDFTeX
  \usepackage[T1]{fontenc}
  \usepackage[utf8]{inputenc}
  \usepackage{textcomp} % provide euro and other symbols
\else % if luatex or xetex
  \usepackage{unicode-math}
  \defaultfontfeatures{Scale=MatchLowercase}
  \defaultfontfeatures[\rmfamily]{Ligatures=TeX,Scale=1}
\fi
\usepackage{lmodern}
\ifPDFTeX\else  
    % xetex/luatex font selection
\fi
% Use upquote if available, for straight quotes in verbatim environments
\IfFileExists{upquote.sty}{\usepackage{upquote}}{}
\IfFileExists{microtype.sty}{% use microtype if available
  \usepackage[]{microtype}
  \UseMicrotypeSet[protrusion]{basicmath} % disable protrusion for tt fonts
}{}
\makeatletter
\@ifundefined{KOMAClassName}{% if non-KOMA class
  \IfFileExists{parskip.sty}{%
    \usepackage{parskip}
  }{% else
    \setlength{\parindent}{0pt}
    \setlength{\parskip}{6pt plus 2pt minus 1pt}}
}{% if KOMA class
  \KOMAoptions{parskip=half}}
\makeatother
\usepackage{xcolor}
\usepackage[top=20mm,left=20mm,heightrounded]{geometry}
\setlength{\emergencystretch}{3em} % prevent overfull lines
\setcounter{secnumdepth}{-\maxdimen} % remove section numbering
% Make \paragraph and \subparagraph free-standing
\ifx\paragraph\undefined\else
  \let\oldparagraph\paragraph
  \renewcommand{\paragraph}[1]{\oldparagraph{#1}\mbox{}}
\fi
\ifx\subparagraph\undefined\else
  \let\oldsubparagraph\subparagraph
  \renewcommand{\subparagraph}[1]{\oldsubparagraph{#1}\mbox{}}
\fi

\usepackage{color}
\usepackage{fancyvrb}
\newcommand{\VerbBar}{|}
\newcommand{\VERB}{\Verb[commandchars=\\\{\}]}
\DefineVerbatimEnvironment{Highlighting}{Verbatim}{commandchars=\\\{\}}
% Add ',fontsize=\small' for more characters per line
\usepackage{framed}
\definecolor{shadecolor}{RGB}{241,243,245}
\newenvironment{Shaded}{\begin{snugshade}}{\end{snugshade}}
\newcommand{\AlertTok}[1]{\textcolor[rgb]{0.68,0.00,0.00}{#1}}
\newcommand{\AnnotationTok}[1]{\textcolor[rgb]{0.37,0.37,0.37}{#1}}
\newcommand{\AttributeTok}[1]{\textcolor[rgb]{0.40,0.45,0.13}{#1}}
\newcommand{\BaseNTok}[1]{\textcolor[rgb]{0.68,0.00,0.00}{#1}}
\newcommand{\BuiltInTok}[1]{\textcolor[rgb]{0.00,0.23,0.31}{#1}}
\newcommand{\CharTok}[1]{\textcolor[rgb]{0.13,0.47,0.30}{#1}}
\newcommand{\CommentTok}[1]{\textcolor[rgb]{0.37,0.37,0.37}{#1}}
\newcommand{\CommentVarTok}[1]{\textcolor[rgb]{0.37,0.37,0.37}{\textit{#1}}}
\newcommand{\ConstantTok}[1]{\textcolor[rgb]{0.56,0.35,0.01}{#1}}
\newcommand{\ControlFlowTok}[1]{\textcolor[rgb]{0.00,0.23,0.31}{#1}}
\newcommand{\DataTypeTok}[1]{\textcolor[rgb]{0.68,0.00,0.00}{#1}}
\newcommand{\DecValTok}[1]{\textcolor[rgb]{0.68,0.00,0.00}{#1}}
\newcommand{\DocumentationTok}[1]{\textcolor[rgb]{0.37,0.37,0.37}{\textit{#1}}}
\newcommand{\ErrorTok}[1]{\textcolor[rgb]{0.68,0.00,0.00}{#1}}
\newcommand{\ExtensionTok}[1]{\textcolor[rgb]{0.00,0.23,0.31}{#1}}
\newcommand{\FloatTok}[1]{\textcolor[rgb]{0.68,0.00,0.00}{#1}}
\newcommand{\FunctionTok}[1]{\textcolor[rgb]{0.28,0.35,0.67}{#1}}
\newcommand{\ImportTok}[1]{\textcolor[rgb]{0.00,0.46,0.62}{#1}}
\newcommand{\InformationTok}[1]{\textcolor[rgb]{0.37,0.37,0.37}{#1}}
\newcommand{\KeywordTok}[1]{\textcolor[rgb]{0.00,0.23,0.31}{#1}}
\newcommand{\NormalTok}[1]{\textcolor[rgb]{0.00,0.23,0.31}{#1}}
\newcommand{\OperatorTok}[1]{\textcolor[rgb]{0.37,0.37,0.37}{#1}}
\newcommand{\OtherTok}[1]{\textcolor[rgb]{0.00,0.23,0.31}{#1}}
\newcommand{\PreprocessorTok}[1]{\textcolor[rgb]{0.68,0.00,0.00}{#1}}
\newcommand{\RegionMarkerTok}[1]{\textcolor[rgb]{0.00,0.23,0.31}{#1}}
\newcommand{\SpecialCharTok}[1]{\textcolor[rgb]{0.37,0.37,0.37}{#1}}
\newcommand{\SpecialStringTok}[1]{\textcolor[rgb]{0.13,0.47,0.30}{#1}}
\newcommand{\StringTok}[1]{\textcolor[rgb]{0.13,0.47,0.30}{#1}}
\newcommand{\VariableTok}[1]{\textcolor[rgb]{0.07,0.07,0.07}{#1}}
\newcommand{\VerbatimStringTok}[1]{\textcolor[rgb]{0.13,0.47,0.30}{#1}}
\newcommand{\WarningTok}[1]{\textcolor[rgb]{0.37,0.37,0.37}{\textit{#1}}}

\providecommand{\tightlist}{%
  \setlength{\itemsep}{0pt}\setlength{\parskip}{0pt}}\usepackage{longtable,booktabs,array}
\usepackage{calc} % for calculating minipage widths
% Correct order of tables after \paragraph or \subparagraph
\usepackage{etoolbox}
\makeatletter
\patchcmd\longtable{\par}{\if@noskipsec\mbox{}\fi\par}{}{}
\makeatother
% Allow footnotes in longtable head/foot
\IfFileExists{footnotehyper.sty}{\usepackage{footnotehyper}}{\usepackage{footnote}}
\makesavenoteenv{longtable}
\usepackage{graphicx}
\makeatletter
\def\maxwidth{\ifdim\Gin@nat@width>\linewidth\linewidth\else\Gin@nat@width\fi}
\def\maxheight{\ifdim\Gin@nat@height>\textheight\textheight\else\Gin@nat@height\fi}
\makeatother
% Scale images if necessary, so that they will not overflow the page
% margins by default, and it is still possible to overwrite the defaults
% using explicit options in \includegraphics[width, height, ...]{}
\setkeys{Gin}{width=\maxwidth,height=\maxheight,keepaspectratio}
% Set default figure placement to htbp
\makeatletter
\def\fps@figure{htbp}
\makeatother
% definitions for citeproc citations
\NewDocumentCommand\citeproctext{}{}
\NewDocumentCommand\citeproc{mm}{%
  \begingroup\def\citeproctext{#2}\cite{#1}\endgroup}
\makeatletter
 % allow citations to break across lines
 \let\@cite@ofmt\@firstofone
 % avoid brackets around text for \cite:
 \def\@biblabel#1{}
 \def\@cite#1#2{{#1\if@tempswa , #2\fi}}
\makeatother
\newlength{\cslhangindent}
\setlength{\cslhangindent}{1.5em}
\newlength{\csllabelwidth}
\setlength{\csllabelwidth}{3em}
\newenvironment{CSLReferences}[2] % #1 hanging-indent, #2 entry-spacing
 {\begin{list}{}{%
  \setlength{\itemindent}{0pt}
  \setlength{\leftmargin}{0pt}
  \setlength{\parsep}{0pt}
  % turn on hanging indent if param 1 is 1
  \ifodd #1
   \setlength{\leftmargin}{\cslhangindent}
   \setlength{\itemindent}{-1\cslhangindent}
  \fi
  % set entry spacing
  \setlength{\itemsep}{#2\baselineskip}}}
 {\end{list}}
\usepackage{calc}
\newcommand{\CSLBlock}[1]{\hfill\break\parbox[t]{\linewidth}{\strut\ignorespaces#1\strut}}
\newcommand{\CSLLeftMargin}[1]{\parbox[t]{\csllabelwidth}{\strut#1\strut}}
\newcommand{\CSLRightInline}[1]{\parbox[t]{\linewidth - \csllabelwidth}{\strut#1\strut}}
\newcommand{\CSLIndent}[1]{\hspace{\cslhangindent}#1}

\KOMAoption{captions}{tableheading}
\usepackage{wrapfig}
\usepackage{subcaption}
\usepackage{amsmath}
\usepackage{cancel}
\usepackage{hyperref}
\usepackage{tikz}
\usepackage{tabularx}
\renewcommand{\maketitle}{}
\usepackage{fancyhdr}
\pagestyle{fancy}
\fancyhf{}
\fancyhead[L]{\rightmark}
\fancyhead[R]{\thepage}
\fancyfoot[C]{\thepage}
\makeatletter
\@ifpackageloaded{caption}{}{\usepackage{caption}}
\AtBeginDocument{%
\ifdefined\contentsname
  \renewcommand*\contentsname{Table of contents}
\else
  \newcommand\contentsname{Table of contents}
\fi
\ifdefined\listfigurename
  \renewcommand*\listfigurename{Figurliste}
\else
  \newcommand\listfigurename{Figurliste}
\fi
\ifdefined\listtablename
  \renewcommand*\listtablename{Tabelliste}
\else
  \newcommand\listtablename{Tabelliste}
\fi
\ifdefined\figurename
  \renewcommand*\figurename{Figur}
\else
  \newcommand\figurename{Figur}
\fi
\ifdefined\tablename
  \renewcommand*\tablename{Tabell}
\else
  \newcommand\tablename{Tabell}
\fi
}
\@ifpackageloaded{float}{}{\usepackage{float}}
\floatstyle{ruled}
\@ifundefined{c@chapter}{\newfloat{codelisting}{h}{lop}}{\newfloat{codelisting}{h}{lop}[chapter]}
\floatname{codelisting}{Listing}
\newcommand*\listoflistings{\listof{codelisting}{List of Listings}}
\makeatother
\makeatletter
\makeatother
\makeatletter
\@ifpackageloaded{caption}{}{\usepackage{caption}}
\@ifpackageloaded{subcaption}{}{\usepackage{subcaption}}
\makeatother
\ifLuaTeX
  \usepackage{selnolig}  % disable illegal ligatures
\fi
\usepackage{bookmark}

\IfFileExists{xurl.sty}{\usepackage{xurl}}{} % add URL line breaks if available
\urlstyle{same} % disable monospaced font for URLs
\hypersetup{
  colorlinks=true,
  linkcolor={blue},
  filecolor={Maroon},
  citecolor={Blue},
  urlcolor={Blue},
  pdfcreator={LaTeX via pandoc}}

\author{}
\date{}

\begin{document}


\newgeometry{left=0cm, right=0cm, top=0cm, bottom=0cm}
\vspace*{0.5cm} 
\hspace*{1.5cm}\includegraphics[width=10cm]{dokumentobjekter/texstuff/UiT_Logo_Bok_Bla_RGB.png} 


\begin{flushleft}
    \vspace*{0.5cm}
    \hspace*{2.5cm}{\color{black}\fontsize{11}{13.2}\selectfont Handelshøgskolen ved UiT \\[0.2em]
    \hspace*{2.5cm}\color{black}\fontsize{8}{13.2}\selectfont Fakultet for biovitenskap, fiskeri og økonomi \\[0.2em]
    \hspace*{2.5cm}\large{\color{black}\textbf{Mappeoppgave 2}}  \\[0.5em]
    \hspace*{2.5cm}\color{black}\fontsize{12}{14.4}\selectfont Mappeoppgave 2 i: SOK-2030 Næringsøkonomi og konkurransestrategi \\[0.5em]
\hspace*{2.5cm}\color{black}\fontsize{11}{13.2}\selectfont Kandidatnummer: 18 \\[0.5em]
    \hspace*{2.5cm}\color{black}\fontsize{11}{13.2}\selectfont Sok-2030, Vår 2024 \\[0.5em]
    \hspace*{2.0cm}
    \par}
\end{flushleft} 



\begin{tikzpicture}[remember picture, overlay]
    \node[anchor=south west, inner sep=0] at (current page.south west) {\includegraphics[width=\paperwidth]{dokumentobjekter/texstuff/forside_bilde.png}};
\end{tikzpicture}


\newgeometry{left=20mm, right=20mm, top=20mm, bottom=20mm}





\thispagestyle{plain}
\begin{center}
    \Large
    \textbf{Sammendrag}
\end{center}



\newpage
\hypersetup{linkcolor=black}
\renewcommand{\contentsname}{Innholdsfortegnelse}
\renewcommand*{\figureautorefname}{Figur}
\renewcommand*{\tableautorefname}{Tabell}
\tableofcontents
\listoffigures
\listoftables
\hypersetup{linkcolor=blue}
\newpage

\begin{enumerate}
\def\labelenumi{\arabic{enumi}.}
\tightlist
\item
  Oppgave 1 (30\%)
\end{enumerate}

Olivita AS ble etablert i 2002 av to professorer fra Universitetet i
Tromsø (UiT). Selskapet tilbyr kosttilskuddet Olivita, som inneholder
omega-3 og er utviklet for å støtte hjerte, ledd og immunforsvar.
Produktet har vært patentbeskyttet frem til 2023, og Olivita har hatt
eksklusiv rett til produksjon av dette omega-3 produktet. Etter
patentets utløp har det nye selskapet Dr Choice AS kommet på markedet og
tilbyr Easy Choice Omega-3. I markedet for omega-3 produkter vil Olivita
AS fortsette å være en ledende aktør, mens Dr Choice AS vil utfordre som
en nykommer.

I dette marked er det følgende invers etterspørsel \[
P = 990 - \frac{1}{60}(q_O + q_C)
\] hvor \(q_O\) er antall solgte flasker med Olivita, \(q_C\) er antall
solgte flasker Easy Choice Omega-3 og \(P\) er pris per flaske av
Omega-3 produktene. I produksjon av Omega-3 produktene vil begge
bedriftene ha konstante marginalkostnader på kr 50 per produsert flaske.
Faste kostnader for begge bedriftene er på 3 millioner kroner.

\subsection{Optimal tilpasning}\label{optimal-tilpasning}

\subsubsection{a) Hva blir optimal tilpasning i dette markedet når Olivita kan gjøre sine strategiske valg før konkurrenten, Dr Choice AS, gjør sitt valg?}\label{a-hva-blir-optimal-tilpasning-i-dette-markedet-nuxe5r-olivita-kan-gjuxf8re-sine-strategiske-valg-fuxf8r}

For å beregne dette så må jeg først bestemme meg for hvilken modell som
skal brukes ut av de tre modellene som er brukt i kurset: Cournot,
Bertrand og Stackelberg. Ut av de tre modellene så er vil bedriftene i
Cournot og Stackelberg konkurrere på kvantum, mens i Bertrand
konkurrerer de på pris. Det som skiller mellom hva de konkurrerer på er
om det de produserer er strategiske komplimenter eller substitutter.
Disse modellene er da Oligopolmodeller som er en modeller for markeder
med få aktører.

Strategiske komplimenter betyr for eksempel varer som er komplimenter
til hverandre, som for eksempel kaffe og melk. Hvis prisen på kaffe øker
så vil etterspørselen etter melk også øke. Strategiske substitutter er
varer som er substitutter til hverandre, altså varer som er svært like
hverandre som for eksempel Pepsi og Coca Cola. Hvis prisen på Pepsi øker
så vil etterspørselen etter Coca Cola øke.

Da det er sekvensielle valg og de har samme kostnader så bruker jeg
stackelberg modellen som modellerer hvordan aktørene tar valg i en
sekvensiell rekkefølge. Stackelberg modellen er en modell som brukes for
å analysere hvilket kvantum som produseres av hver aktør i et marked når
en aktør tar sitt valg først og den andre aktøren tar sitt valg etterpå.

Grunnen til at jeg bruker en kvantumsmodell er at jeg antar at disse
varene er strategiske substitutter. Dette betyr at hvis en bedrift øker
sin produksjon vil det føre til at konkurransebedriften vil redusere sin
produksjon. Dette vises under AUTOREF HER FOR EN AV DE NEDRE SIDENE av
reaksjonsfunksjonen til bedriftene.

Denne modellen løses i to trinn der lederbedriften Olivita velger sitt
kvantum i trinn 1, men modellen løses baklengs ved at vi først løser
trinn 2. Grunnen til dette er at Olivita vil finne ut Dr Choice AS sin
beste reaksjonsfunksjon til enhver mengde Olivita produserer. Dette gjør
at Olivita kan maksimere sin egen profitt ved å velge kvantum som gir
høyest profitt gitt Dr Choice AS sin reaksjon. (Pepall et al., 2014, s.
265-268)

Vi har da profittfunksjonene for begge bedriftene ved å ta markedspris
per enhet minus marginalkostnad, multiplisert med kvantum solgt, minus
faste kostnader.

\begin{align*}
\pi_O=(P-50) \cdot q_O-3000000 \\
\pi_C=(P-50) \cdot q_C-3000000
\end{align*}

Begge bedriftene har samme marginalkostnader og faste kostnader så jeg
skriver disse som en felles profitt funksjon. Jeg gjør det også på
generell form da utskriftene fra utregning blir rotete.

\[
\pi = Q_O \cdot (a - b \cdot (Q_O + Q_C) - c) 
\] Som sagt skal trinn 2 løses først så ved å derivere profitten til Dr
Choice AS med hensyn på \(Q_C\) får jeg:

\[
\frac{\partial \pi}{\partial Q_C} = a- 2bQ_C - bQ_O -c
\] Når denne nå settes lik null får jeg reaksjons funksjonen til Dr
Choice AS.

\[
RF2 = Q_C= \frac{a-bQ_O - c}{2b}
\]

Denne settes så inn i den deriverte profittfunksjonen til Olivita og
løses for \(Q_O\).

\begin{align*}
Q_O = a-\frac{b Q_O}{2} - b \left(q_O + \frac{2-b q_O - c}{2b}\right)-c
\end{align*}

Denne løses for \(Q_O\) for å finne optimalt kvantum for Olivita som
velger sitt kvantum først. \[
RF1 = Q_O = \frac{a-c}{2b}
\]

Vi setter den så inn i reaksjonsfunksjonen til Dr Choice AS for å finne
optimalt kvantum for Dr Choice AS.

\[
Q_C = \frac{\frac{a}{2}-\frac{c}{2}}{2b} = \frac{a-c}{4b}
\] Da Q = \(Q_O + Q_C\) så kan vi finne optimalt kvantum for begge
bedriftene og prisen i markedet.

\begin{align*}
P &= a - b(Q_O + Q_C) \\
&= a- b(\frac{a-c}{2b} + \frac{a- c}{4b}) \\
&= \frac{a}{4} + \frac{c}{4}
\end{align*}

Vi kan nå finne markedsprisen ved å sette inn kvantum for begge
bedriftene i den inverse etterspørselen.

$\displaystyle 285.0$

Og kvantumet til begge blir

\begin{verbatim}
Kvantum for Olivitas: 28200 og Kvantum for Dr Choice AS: 14100. 
Det totale kvantumet i markedet er 42300 til en pris på 285.
\end{verbatim}

Og profitten til begge bedriftene

$\displaystyle π_{O} = 3627000.0$

$\displaystyle π_{C} = 313500.0$

\newpage

\subsubsection{b) Vil det være en fordel for Olivita å ha mulighet til å
gjøre sine valg før konkurrenten gjør sitt
valg?}\label{b-vil-det-vuxe6re-en-fordel-for-olivita-uxe5-ha-mulighet-til-uxe5-gjuxf8re-sine-valg-fuxf8r-konkurrenten-gjuxf8r-sitt-valg}

Ja det vil være en fordel for Olivita men for å bekrefte dette så vil
jeg beregne hvilket kvantum de ville valgt i en vanlig cournot modell
uten sekvensiell valg.

Jeg gjøre dette igjen på generell form for å så sette inn tall. \[
P = a - b(Q_O + Q_C)
\] Marginalkostnaden til begge bedriftene er konstant og lik og for å
finne etterspørselsfunksjonen for hver bedrift betraktes produsert
mengde for den andre bedrift som konstant.

Etterspørselen til Dr Choice AS er gitt ved: \[
P_C = (a  + bQ_O)-bQ_C
\] For å finne optimalt kvantum for Dr Choice AS så derivertes profitten
til Dr Choice AS med hensyn på \(Q_C\) for å finne marginalinnekten.

\begin{align*}
\pi_C = (P_C - c)Q_C = (a  + bQ_O - bQ_C - c)Q_C \\
\frac{\pi_C}{\partial Q_C} = (a - bQ_O) - 2bQ_C
\end{align*}

Dette gir oss optimalt kvantum for Dr Choice AS ved å sette lik marginal
kostnad. \[
Q_C = \frac{a - c}{2b} - \frac{Q_O}{2}
\]

Vi kan så finne reaksjonsfunksjonen til Olivita ved å sette dette
kvantumet inn i etterspørselen til Olivita. \[
Q_O = \frac{a - c}{2b} - \frac{Q_C}{2}
\] Vi løser denne ved å sette inn for \(Q_C\) \[
Q_O = \frac{a - c}{2b} - \left(\frac{\frac{a - c}{2b} - \frac{Q_O}{2}}{2}\right)
\] Som endelig gir oss \[
Q_O = \frac{a - c}{3b} = Q_C
\]

Vi kan nå finne markedsprisen ved å sette inn kvantum for begge
bedriftene i den inverse etterspørselen.

\begin{verbatim}
Optimalt kvantum for Q_O er 18800 og for Q_C er 18800. 
Så det totale kvantumet blir 37600 og markedsprisen blir 363.333
Profit for Q_O:  2060363.33 Profit for Q_C:  2060363.33 
Total profit:  4120726.66
\end{verbatim}

Så vi kan se at dersom Olivita ikke får gjøre sitt valg først så vil det
totale kvantumet i markedet være 37600 som er lavere enn det totale
kvantumet under stackelberg på 42300. Det er derimot en høyere pris på
363 istedet for 285.

\subsubsection{Begrunnelse for valg av
modell}\label{begrunnelse-for-valg-av-modell}

$\displaystyle 28200.0 - 0.5 Q_{C}$

$\displaystyle 28200.0 - 0.5 Q_{O}$

Her har jeg tatt å satt inn for verdiene til \(a\), \(b\) og \(c\) i
reaksjonsfunksjonene til bedriftene. Det vi kan se er at begge
reaksjonsfunksjonene har negativ helning så dersom Olivita øker sin
produksjon så vil reaksjonsfunksjonen til Dr Choice AS reduseres og det
samme vil skje dersom Dr Choice AS øker sin produksjon. Dette tyder på
at det er strategiske substitutter.

kanskje nevne nash likevekt

\clearpage

\subsection{Optimal tilpasning før
fusjon}\label{optimal-tilpasning-fuxf8r-fusjon}

\begin{enumerate}
\def\labelenumi{\arabic{enumi}.}
\setcounter{enumi}{1}
\tightlist
\item
  Oppgave 2 (70\%)
\end{enumerate}

Markedet for produksjon av mikroøl består av tre lokale bryggerier:
Graff Brygghus, Bryggeri 13 og Mack Mikrobryggeri. Etterspørselen i
dette markedet er gitt ved: \[
P = 175-4Q
\] hvor \(P\) er markedspris per flaske mikroøl, \(Q\) er totalt kvantum
(antall tusen flasker), som er summen av produksjonen til de tre
bryggeriene: \(Q = q_G + q_B + q_M\), der \(q_G\) er produsert kvantum
for Graff Brygghus, \(q_B\) er produsert kvantum for Bryggeri 13 og
\(q_M\) er produsert kvantum for Mack Mikrobryggeri.

Mack Mikrobryggeri, som er en del av Mack Ølbryggeri, har en mer
effektiv produksjonslinje enn de to andre, med konstante
marginalkostnader på 7 kr per flaske, mens Graff Brygghus og Bryggeri 13
har marginalkostnader på 10 kr per flaske. Alle tre mikrobryggeriene har
faste årlige kostnader på 300 000 kr. Styrene i selskapene Mack
Mikrobryggeri og Bryggeri 13 har startet samtaler knyttet til mulig
fusjon av disse to selskapene. Ved en fusjon vil all produksjon flyttes
til Mack Mikrobryggeri. De faste kostnadene vil også reduseres ved
sammenslåing av selskapene, og totalt utgjøre kr 500 000 per år for det
fusjonerte selskapet.

\subsubsection{a) Vil en slik fusjon være lønnsom for de fusjonerte
partene?}\label{a-vil-en-slik-fusjon-vuxe6re-luxf8nnsom-for-de-fusjonerte-partene}

I en slik situasjon så starter jeg med å beregne hvordan markedet er før
en fusjon ved bruk av Cournot modellen for bedrifter med asymmetriske
bedrifter siden en av de tre bedriftene har forskjellige kostnader. Da
varene er homogene eller veldig like så vil vi anta at det er samme
kundebase for alle tre og at bedriftene konkurrerer med kvantitet.

\paragraph{Cournot modellen før
fusjon}\label{cournot-modellen-fuxf8r-fusjon}

For å finne likevekten i markedet før fusjon så starter vi med å finne
reaksjonsfunksjonene til bedriftene og jeg gjør dette igjen på generell
form for å gjøre det ryddigere og lettere å forstå.

Jeg skriver da igjen etterspørselsfunksjonen som: \[
P = a - bQ
\] \(Q\) er igjen lik \(\Sigma q_i\) og vi har at
\(Q = q_G + q_B + q_M\). der \(q_G\) er for Graff Brygghus, \(q_B\) er
for Bryggeri 13 og \(q_M\) er for Mack Mikrobryggeri. Inntektene til
bedriftene er gitt ved \(I\): \[
I_i = P \cdot q_i\cdot(a-b\cdot q_i)
\] og kostnader er gitt ved: \[
c_i = q_i \cdot c_i + f_i
\]

Dette kan settes sammen for å gi oss profittfunksjonen til bedriftene:

\begin{align*}
\pi_G &= -c_G\cdot q_G - f_G + (a-b\cdot(q_G+q_B+q_M))\cdot q_G \\
\pi_B &= -c_B\cdot q_B - f_B + (a-b\cdot(q_G+q_B+q_M))\cdot q_B \\
\pi_M &= -c_M\cdot q_M - f_M + (a-b\cdot(q_G+q_B+q_M))\cdot q_M
\end{align*}

Jeg deriverer så disse og løser for \(q_i\) for å finne
reaksjonsfunksjonene til bedriftene og setter inn verdier for \(a\),
\(b\) og \(c_i\) for å finne kvantumene som produseres av bedriftene.

\begin{align*}
q_B &= \frac{a-3\cdot c_B + c_G + c_M}{4\cdot b}= \frac{175-3\cdot 10 + 10 + 7}{4\cdot 4} = 10.125 \\
q_G &= \frac{a + c_B - 3\cdot c_G + c_M}{4\cdot b} =  \frac{175 + 10 - 3\cdot 10 + 7}{4\cdot 4} = 10.125 \\
q_M &= \frac{a + c_B + c_G -3\cdot c_M}{4\cdot b} = \frac{175 + 10 + 10 -3\cdot 7}{4\cdot 4} = 10.875
\end{align*}

Dette gir oss at Graff Brygghus og Bryggeri 13 produserer 10.125 tusen
flasker mikroøl, mens Mack Mikrobryggeri produserer 10.875 tusen flasker
mikroøl.

\begin{figure}[h]
\centering
\includegraphics[width=0.9\textwidth]{dokumentobjekter/figurer/markedsandel_mikrobryggerier.png}
\caption{Markedsandelen til mikrobryggeriene}
\label{fig:markedsandel}
\end{figure}

For å nå finne markedsprisen så setter jeg dette inn i
etterspørselsfunksjonen og får: \[
P = 175 - 4\cdot(10.125 + 10.125 + 10.875) = 175 - 4\cdot 31.125 = 175 - 124.5 = 50.5
\] Dette gir oss at markedsprisen er 50.5 kr per flaske mikroøl. Så kan
jeg bare sette inn tallene i de tidligere profittfunksjonene og får:

\begin{align*}
\pi_G = 110.0625 \\
\pi_B = 110.0625 \\
\pi_M = 173.0625
\end{align*}

Da tallene er gitt i tusen så gir dette oss at Graff Brygghus og
Bryggeri 13 hadde hver en profitt på 1100625 kr, mens Mack Mikrobryggeri
har en profitt på 1730625 kr.

Nå starter jeg ved å se på hva som skjer om Mack Mikrobryggeri og
Bryggeri 13 fusjonerer.

\clearpage

\subsection{Optimal tilpasning etter
fusjon}\label{optimal-tilpasning-etter-fusjon}

Når Mack Mikrobryggeri og Bryggeri 13 fusjonerer vil vi ha et marked med
2 aktører med asymmetriske kostnader så da bruker jeg igjen modellen for
Cournot-konkurranse med asymmetriske bedrifter men nå med 2 aktører. Jeg
blir å betegne Mack og Bryggeri 13 \(BM\). Jeg setter opp ligningene for
profitten til hver av bedriftene etter fusjonen:

\begin{align*}
\pi_G = -cG \cdot q_G + f_G + q_G \cdot (a - b \cdot (q_G + q_{BM}))\\
\pi_BM = -c_{BM} \cdot q_{BM} - f_{BM} + q_{BM} \cdot (a - b \cdot (q_G + q_{BM}))
\end{align*}

Jeg deriverer igjen og løser for \(q_G\) og \(q_{BM}\) for å finne
reaksjonsfunksjonene og kvantum:

\begin{align*}
q_G = \frac{a-2\cdot c_G + c_{BM}}{3\cdot b} = \frac{175-2\cdot 10 + 7}{3\cdot 4} =  13.5\\
q_{BM} = \frac{a + c_G -2\cdot c_{BM}}{3\cdot b} = \frac{175 + 10 -2\cdot 7}{3\cdot 4} = 14.25
\end{align*}

Ved å substituere inn tallene får vi at Mack Mikrobryggeri vil produsere
13.5 tusen flasker mikroøl, mens Bryggeri 13 vil produsere 14.25 tusen
flasker mikroøl. Jeg kan nå sette dette inn i etterspørselsfunksjonen å
få inn markedspris \[
p = 175 - 4 \cdot (13.5 + 14.25) = 64
\] Og vi kan se denne er høyere enn den opprinnelige prisen før fusjon.
Jeg kan nå sette dette inn i profittfunksjonene for å finne profitten
til hver av bedriftene etter fusjonen.

\begin{align*}
\pi_G = 429 \\
\pi_{BM} = 312.25
\end{align*}

Vi kan se at Graff har tjent mer på fusjonen enn Mack og Bryggeri 13. Vi
kan også se hvordan dette har skjedd i \autoref{fig:markedsandel_etter}
på neste side

\newpage

\begin{figure}[t]
\centering
\includegraphics[width=0.7\textwidth]{dokumentobjekter/figurer/markedsandel_mikrobryggerier_fusjon.png}
\caption{Markedsandelen til mikrobryggeriene etter fusjonen}
\label{fig:markedsandel_etter}
\end{figure}

Fusjonsparadokset er at fusjonen har gitt økt profitt for Graff
Brygghus,

blablabla

I sum. Fusjonen har gitt økt profitt for begge selskapene og selv om det
har vært bedre for Graff Brygghus så vil Mack og Bryggeri 13 også tjene
mer på fusjonen igjennom økte priser. Av den grunn vil fusjonen være en
god idé for begge selskapene.

\clearpage

Videre i oppgaven skal vi anta at fusjon mellom Mack Mikrobryggeri og
Bryggeri 13 blir gjennomført, og det nye selskapet vil operere under
navnet Mack Mikrobrygg 13. Markedet for produksjon av mikroøl vil da
bestå av to lokale produsenter: Mack Mikrobrygg 13 og Graff Brygghus.
For å styrke sin posisjon i markedet, investerer Graff Brygghus i nytt
og mer effektivt produksjonsutstyr, noe som reduserer deres variable
kostnader til kr 7 per flaske. Denne investeringen vil gi selskapet økte
faste kostnader på kr 200.000. Totale faste kostnader for begge
bryggeriene er da på kr 500.000 for hvert av selskapene.

I restaurantbransjen i Tromsø er Restaurant Gruppen Holdig (RGH) en
sentral aktør, som har monopol i sitt segment. RGH kjøper sitt mikroøl
fra de to lokale produsentene Mack Mikrobrygg 13 og Graff Brygghus. For
å drifte sine restauranter har RGH faste kostnader på kr 600.000.

Etterspørselen etter mikroøl i restaurantbransjen er lik: \[
P = 175 - 2Q
\] hvor \(Q\) er antall solgte flasker mikroøl (antall tusen flasker)
for RGH og \(P\) er prisen for en flaske mikroøl til sluttbruker. For å
ytterligere styrke sin posisjon i oppstrømsmarkedet, vurderer ledelsen i
Mack Mikrobrygg 13 en fusjon med konkurrenten Graff Brygghus. Det antas
at denne fusjonen ikke vil resultere i kostnadsbesparelser for
bryggeriene. Som konsulent for styret i Mack Mikrobrygg 13, er du bedt
om å analysere markedskonsekvensene av en potensiell fusjon mellom Mack
Mikrobrygg 13 og Graff Brygghus. Analysen skal omfatte en vurdering av
dagens markedstilpasning og en sammenligning med tilpasningen etter en
eventuell fusjon i oppstrømsmarkedet.

\subsubsection{b) Basert på din analyse, vil du anbefale styret i Mack
Mikrobrygg 13 å gjennomføre fusjon med Graff
Brygghus?}\label{b-basert-puxe5-din-analyse-vil-du-anbefale-styret-i-mack-mikrobrygg-13-uxe5-gjennomfuxf8re-fusjon-med-graff-brygghus}

Det er igjen snakk om horisontal fusjon men denne gangen vil vi gå fra å
ha 2 aktører til å ha en monopolist. Med etterspørselsfunksjonen gitt
som \(P=175-2Q\) kan vi først beregne markedstilpasningen da de nå har
like marginalkostnader og faste kostnader.

\begin{align*}
\pi_G &= -c_G\cdot q_G - f_G + (a-b\cdot(q_G+q_BM))\cdot q_G \\
\pi_BM &= -c_B\cdot q_BM - f_BM + (a-b\cdot(q_G+q_BM))\cdot q_BM \\
\end{align*}

Derivert med hensyn på \(q_G\) og \(q_BM\) får vi:

\begin{align*}
a-b\cdot q_C - b(q_C + q_{BM})- c_G \\
a-b\cdot q_{BM} - b(q_G + q_{BM})- c_{BM} 
\end{align*}

Løst for \(q_G\) og \(q_{BM}\) får vi:

\begin{align*}
q_G = \frac{a-2\cdot c_G +c_{BM}}{3b} \\
q_BM = \frac{a +c_G - 2\cdot c_{BM}}{3b}
\end{align*}

her starter slurvet for real

dette gir oss kvantum på begge på 28, markedspris på 63 og profitt hver
på 1068.

Etter fusjonen vil vi ha en monopolist og vi kan da regne ut profitten
til monopolisten.

\[
\pi_BMG = -c_{BMG}\cdot q_{BMG} - f_{BMG} + q_{BMG}\cdot (a-b\cdot q_{BMG})
\] derivert og løst for q\_BMG \[
q_{BMG} = \frac{a-c_{BMG}}{2b}
\] totalt kvantum på 42, pris 91 og total profitt 3028.

\clearpage

\subsection{c) Hva blir de samfunnsøkonomiske konsekvensene av en fusjon
mellom Mack Mikrobrygg 13 og Graff
Brygghus?}\label{c-hva-blir-de-samfunnsuxf8konomiske-konsekvensene-av-en-fusjon-mellom-mack-mikrobrygg-13-og-graff-brygghus}

\clearpage

\section{Referanser}\label{referanser}

\phantomsection\label{refs}
\begin{CSLReferences}{1}{0}
\bibitem[\citeproctext]{ref-pepall2014}
Pepall, L., Richards, D. J. \& Norman, G. (2014). \emph{Industrial
organization: Contemporary theory and empirical applications} (Fifth
edition). Wiley.

\end{CSLReferences}

\clearpage

\appendix

\section {Appendix Generell KI bruk}

I løpet av koden så kan det ses mange \# kommentarer der det er skrevet
for eks ``\#fillbetween q1 and q2''. Når jeg skriver kode i Visual
Studio Code så har jeg en plugin som heter Github Copilot. Når jeg
skriver slike kommentarer så kan den foresøke å fullføre kodelinjene
mens jeg skriver de. Noen ganger klarer den det, men andre ikke. Det er
vanskelig å dokumentere hvert bruk der den er brukt siden det ``går
veldig fort'' men siden jeg ikke har fått på plass en slik dokumentasjon
så kan all python kode der det er brukt kommentarer antas som at det er
brukt Github Copilot. Nærmere info om dette KI verktøyet kan ses på
\url{https://github.com/features/copilot}

\clearpage

\section {Appendix Python kode Oppgave 1a}

\begin{Shaded}
\begin{Highlighting}[]
\NormalTok{q\_O, q\_C, p, c, f, pi\_C, pi\_O}\OperatorTok{=}\NormalTok{ sp.symbols(}\StringTok{\textquotesingle{}q\_O q\_C p c f }\CharTok{\textbackslash{}u03C0}\StringTok{\_C }\CharTok{\textbackslash{}u03C0}\StringTok{\_O\textquotesingle{}}\NormalTok{)}

\NormalTok{Invers\_etterspors }\OperatorTok{=}\NormalTok{ sp.Eq(p, (}\DecValTok{990}\OperatorTok{{-}}\NormalTok{(}\DecValTok{1}\OperatorTok{/}\DecValTok{60}\NormalTok{)}\OperatorTok{*}\NormalTok{(q\_C}\OperatorTok{+}\NormalTok{q\_O)))}


\NormalTok{profitt\_1\_eq }\OperatorTok{=}\NormalTok{ sp.Eq(pi\_O, (Invers\_etterspors.rhs}\OperatorTok{{-}}\DecValTok{50}\NormalTok{)}\OperatorTok{*}\NormalTok{q\_C}\OperatorTok{{-}}\DecValTok{3000000}\NormalTok{)}
\NormalTok{profitt\_2\_eq }\OperatorTok{=}\NormalTok{ sp.Eq(pi\_C, (Invers\_etterspors.rhs}\OperatorTok{{-}}\DecValTok{50}\NormalTok{)}\OperatorTok{*}\NormalTok{q\_O}\OperatorTok{{-}}\DecValTok{3000000}\NormalTok{)}


\NormalTok{derivert\_profitt\_2 }\OperatorTok{=}\NormalTok{ sp.diff(profitt\_2\_eq.rhs, q\_O)}

\NormalTok{reaksjon\_olivitas }\OperatorTok{=}\NormalTok{ sp.solve(derivert\_profitt\_2, q\_O)[}\DecValTok{0}\NormalTok{]}

\NormalTok{profitt\_1\_eq }\OperatorTok{=}\NormalTok{ profitt\_1\_eq.subs(q\_O, reaksjon\_olivitas)}

\NormalTok{derivert\_profitt1 }\OperatorTok{=}\NormalTok{ sp.diff(profitt\_1\_eq.rhs, q\_C) }\CommentTok{\#kvantum for olivitas}

\NormalTok{kvantum\_olivitas }\OperatorTok{=}\NormalTok{ sp.solve(derivert\_profitt1, q\_C)[}\DecValTok{0}\NormalTok{]}

\CommentTok{\#setter inn kvantum i reaksjonsfunksjonen til Dr Choice AS}
\NormalTok{kvantum\_choice }\OperatorTok{=}\NormalTok{ reaksjon\_olivitas.subs(q\_C, kvantum\_olivitas)}\CommentTok{\#Kvantum for choice}
\end{Highlighting}
\end{Shaded}

\begin{Shaded}
\begin{Highlighting}[]
\NormalTok{markedspris }\OperatorTok{=}\NormalTok{ Invers\_etterspors.rhs.subs(\{q\_C:kvantum\_olivitas, q\_O:kvantum\_choice\})}
\NormalTok{markedspris}
\end{Highlighting}
\end{Shaded}

$\displaystyle 285.0$

\begin{Shaded}
\begin{Highlighting}[]
\BuiltInTok{print}\NormalTok{(}\SpecialStringTok{f\textquotesingle{}\textquotesingle{}\textquotesingle{}Kvantum for Olivitas: }\SpecialCharTok{\{}\BuiltInTok{round}\NormalTok{(kvantum\_olivitas)}\SpecialCharTok{\}}\SpecialStringTok{ og Kvantum for Dr Choice AS: }\SpecialCharTok{\{}\BuiltInTok{round}\NormalTok{(kvantum\_choice)}\SpecialCharTok{\}}\SpecialStringTok{\textquotesingle{}\textquotesingle{}\textquotesingle{}}\NormalTok{)}
\end{Highlighting}
\end{Shaded}

\begin{verbatim}
Kvantum for Olivitas: 28200 og Kvantum for Dr Choice AS: 14100
\end{verbatim}

\begin{Shaded}
\begin{Highlighting}[]
\NormalTok{profitt\_olivitas }\OperatorTok{=}\NormalTok{ profitt\_1\_eq.rhs.subs(q\_C, kvantum\_olivitas).subs(q\_O, kvantum\_choice)}
\NormalTok{profitt\_olivitas }\OperatorTok{=}\NormalTok{ sp.Eq(pi\_O, profitt\_olivitas)}
\NormalTok{profitt\_olivitas}
\end{Highlighting}
\end{Shaded}

$\displaystyle π_{O} = 3627000.0$

\begin{Shaded}
\begin{Highlighting}[]
\NormalTok{profitt\_choice }\OperatorTok{=}\NormalTok{ profitt\_2\_eq.rhs.subs(q\_O, kvantum\_choice).subs(q\_C, kvantum\_olivitas)}
\NormalTok{profitt\_choice }\OperatorTok{=}\NormalTok{ sp.Eq(pi\_C, profitt\_choice)}
\NormalTok{profitt\_choice}
\end{Highlighting}
\end{Shaded}

$\displaystyle π_{C} = 313500.0$

\clearpage

\section {Appendix Python kode Oppgave 1 b}

\begin{Shaded}
\begin{Highlighting}[]
\ImportTok{import}\NormalTok{ sympy }\ImportTok{as}\NormalTok{ sp}
\NormalTok{a, b, Q\_O, Q\_C, c, f, p, Q }\OperatorTok{=}\NormalTok{ sp.symbols(}\StringTok{\textquotesingle{}a b Q\_O Q\_C c f p Q\textquotesingle{}}\NormalTok{)}


\NormalTok{demand }\OperatorTok{=}\NormalTok{ sp.Eq(p, a}\OperatorTok{{-}}\NormalTok{b}\OperatorTok{*}\NormalTok{(Q\_O}\OperatorTok{+}\NormalTok{Q\_C))}

\NormalTok{income }\OperatorTok{=}\NormalTok{ sp.Eq(p}\OperatorTok{*}\NormalTok{Q, Q}\OperatorTok{*}\NormalTok{(a}\OperatorTok{{-}}\NormalTok{b}\OperatorTok{*}\NormalTok{Q))}
\NormalTok{costs }\OperatorTok{=}\NormalTok{ Q}\OperatorTok{*}\NormalTok{c}\OperatorTok{{-}}\NormalTok{f}

\NormalTok{MR }\OperatorTok{=}\NormalTok{ sp.solve(sp.diff(income.rhs}\OperatorTok{{-}}\NormalTok{costs, Q), c)[}\DecValTok{0}\NormalTok{]}

\NormalTok{MR1 }\OperatorTok{=}\NormalTok{ sp.Eq(sp.diff((a}\OperatorTok{{-}}\NormalTok{b}\OperatorTok{*}\NormalTok{(Q\_O}\OperatorTok{+}\NormalTok{Q\_C))}\OperatorTok{*}\NormalTok{Q\_O}\OperatorTok{{-}}\NormalTok{f, Q\_O), c)}
\NormalTok{RF1  }\OperatorTok{=}\NormalTok{ sp.solve(MR1, Q\_O)[}\DecValTok{0}\NormalTok{]}

\NormalTok{MR2 }\OperatorTok{=}\NormalTok{ sp.Eq(sp.diff((a}\OperatorTok{{-}}\NormalTok{b}\OperatorTok{*}\NormalTok{(Q\_C}\OperatorTok{+}\NormalTok{Q\_O))}\OperatorTok{*}\NormalTok{Q\_C}\OperatorTok{{-}}\NormalTok{f, Q\_C), c)}
\NormalTok{RF2 }\OperatorTok{=}\NormalTok{ sp.solve(MR2, Q\_C)[}\DecValTok{0}\NormalTok{]}


\NormalTok{optimalt\_kvantum1 }\OperatorTok{=}\NormalTok{ sp.solve(sp.Eq(RF1.subs(Q\_C, RF2), Q\_O))[}\DecValTok{0}\NormalTok{][Q\_O]}

\NormalTok{optimalt\_kvantum2 }\OperatorTok{=}\NormalTok{ sp.solve(sp.Eq(RF2.subs(Q\_O, RF1), Q\_C))[}\DecValTok{0}\NormalTok{][Q\_C]}

\NormalTok{profitt\_1 }\OperatorTok{=}\NormalTok{ (demand.rhs}\OperatorTok{{-}}\NormalTok{costs).subs(Q, optimalt\_kvantum1)}


\NormalTok{markedspris }\OperatorTok{=} \BuiltInTok{round}\NormalTok{(}\BuiltInTok{float}\NormalTok{(sp.solve(demand.subs(\{a:}\DecValTok{990}\NormalTok{, b:}\DecValTok{1}\OperatorTok{/}\DecValTok{60}\NormalTok{, Q\_O:optimalt\_kvantum1, Q\_C:optimalt\_kvantum2, c: }\DecValTok{50}\NormalTok{\}), p)[}\DecValTok{0}\NormalTok{]),}\DecValTok{3}\NormalTok{)}

\NormalTok{profitt\_1 }\OperatorTok{=}\NormalTok{ (demand.rhs}\OperatorTok{{-}}\NormalTok{costs).subs(\{Q\_O: optimalt\_kvantum1, Q\_C: optimalt\_kvantum2, Q: optimalt\_kvantum1, f: }\DecValTok{3000000}\NormalTok{, a: }\DecValTok{990}\NormalTok{, b: }\DecValTok{1}\OperatorTok{/}\DecValTok{60}\NormalTok{, c: }\DecValTok{50}\NormalTok{\}) }
\NormalTok{profitt\_2 }\OperatorTok{=}\NormalTok{ (demand.rhs}\OperatorTok{{-}}\NormalTok{costs).subs(\{Q\_O: optimalt\_kvantum1, Q\_C: optimalt\_kvantum2, Q: optimalt\_kvantum2, f: }\DecValTok{3000000}\NormalTok{, a: }\DecValTok{990}\NormalTok{, b: }\DecValTok{1}\OperatorTok{/}\DecValTok{60}\NormalTok{, c: }\DecValTok{50}\NormalTok{\})}



\NormalTok{kvantum\_QO }\OperatorTok{=} \BuiltInTok{int}\NormalTok{(optimalt\_kvantum1.subs(\{a:}\DecValTok{990}\NormalTok{, b:}\DecValTok{1}\OperatorTok{/}\DecValTok{60}\NormalTok{, c:}\DecValTok{50}\NormalTok{, f:}\DecValTok{3000000}\NormalTok{\}))}
\NormalTok{kvantum\_QC }\OperatorTok{=} \BuiltInTok{int}\NormalTok{(optimalt\_kvantum2.subs(\{a:}\DecValTok{990}\NormalTok{, b:}\DecValTok{1}\OperatorTok{/}\DecValTok{60}\NormalTok{, c:}\DecValTok{50}\NormalTok{, f:}\DecValTok{3000000}\NormalTok{\}))}

\BuiltInTok{print}\NormalTok{(}\SpecialStringTok{f\textquotesingle{}\textquotesingle{}\textquotesingle{}Optimalt kvantum for Q\_O er }\SpecialCharTok{\{}\NormalTok{kvantum\_QO}\SpecialCharTok{\}}\SpecialStringTok{ og for Q\_C er }\SpecialCharTok{\{}\NormalTok{kvantum\_QC}\SpecialCharTok{\}}\SpecialStringTok{. Så det totale kvantumet blir }\SpecialCharTok{\{}\NormalTok{(kvantum\_QO}\OperatorTok{+}\NormalTok{kvantum\_QC)}\SpecialCharTok{\}}\SpecialStringTok{ og markedsprisen blir }\SpecialCharTok{\{}\NormalTok{markedspris}\SpecialCharTok{\}}\SpecialStringTok{\textquotesingle{}\textquotesingle{}\textquotesingle{}}\NormalTok{)}

\BuiltInTok{print}\NormalTok{(}\StringTok{\textquotesingle{}\textquotesingle{}\textquotesingle{}Profit for Q\_O: \textquotesingle{}\textquotesingle{}\textquotesingle{}}\NormalTok{, }\BuiltInTok{round}\NormalTok{(}\BuiltInTok{float}\NormalTok{(profitt\_1),}\DecValTok{2}\NormalTok{), }\StringTok{\textquotesingle{}\textquotesingle{}\textquotesingle{}Profit for Q\_C: \textquotesingle{}\textquotesingle{}\textquotesingle{}}\NormalTok{, }\BuiltInTok{round}\NormalTok{(}\BuiltInTok{float}\NormalTok{(profitt\_2),}\DecValTok{2}\NormalTok{), }\StringTok{\textquotesingle{}\textquotesingle{}\textquotesingle{}}\CharTok{\textbackslash{}n}\StringTok{Total profit: \textquotesingle{}\textquotesingle{}\textquotesingle{}}\NormalTok{, }\BuiltInTok{round}\NormalTok{(}\BuiltInTok{float}\NormalTok{(profitt\_1),}\DecValTok{2}\NormalTok{)}\OperatorTok{+}\BuiltInTok{round}\NormalTok{(}\BuiltInTok{float}\NormalTok{(profitt\_2),}\DecValTok{2}\NormalTok{))}
\end{Highlighting}
\end{Shaded}

\begin{verbatim}
Optimalt kvantum for Q_O er 18800 og for Q_C er 18800. Så det totale kvantumet blir 37600 og markedsprisen blir 363.333
Profit for Q_O:  2060363.33 Profit for Q_C:  2060363.33 
Total profit:  4120726.66
\end{verbatim}

\clearpage

\section {Appendix Python kode Oppgave 2 a}

\clearpage

\section {Appendix Python kode Oppgave 2 b}

\clearpage

\section {Appendix Python kode Oppgave 2 c}



\end{document}
