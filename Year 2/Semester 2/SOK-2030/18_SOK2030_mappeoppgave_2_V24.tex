% Options for packages loaded elsewhere
\PassOptionsToPackage{unicode}{hyperref}
\PassOptionsToPackage{hyphens}{url}
\PassOptionsToPackage{dvipsnames,svgnames,x11names}{xcolor}
%
\documentclass[
  12pt,
  a4paper,
  DIV=11,
  numbers=noendperiod]{scrartcl}

\usepackage{amsmath,amssymb}
\usepackage{iftex}
\ifPDFTeX
  \usepackage[T1]{fontenc}
  \usepackage[utf8]{inputenc}
  \usepackage{textcomp} % provide euro and other symbols
\else % if luatex or xetex
  \usepackage{unicode-math}
  \defaultfontfeatures{Scale=MatchLowercase}
  \defaultfontfeatures[\rmfamily]{Ligatures=TeX,Scale=1}
\fi
\usepackage{lmodern}
\ifPDFTeX\else  
    % xetex/luatex font selection
\fi
% Use upquote if available, for straight quotes in verbatim environments
\IfFileExists{upquote.sty}{\usepackage{upquote}}{}
\IfFileExists{microtype.sty}{% use microtype if available
  \usepackage[]{microtype}
  \UseMicrotypeSet[protrusion]{basicmath} % disable protrusion for tt fonts
}{}
\makeatletter
\@ifundefined{KOMAClassName}{% if non-KOMA class
  \IfFileExists{parskip.sty}{%
    \usepackage{parskip}
  }{% else
    \setlength{\parindent}{0pt}
    \setlength{\parskip}{6pt plus 2pt minus 1pt}}
}{% if KOMA class
  \KOMAoptions{parskip=half}}
\makeatother
\usepackage{xcolor}
\usepackage[top=20mm,left=20mm,heightrounded]{geometry}
\setlength{\emergencystretch}{3em} % prevent overfull lines
\setcounter{secnumdepth}{-\maxdimen} % remove section numbering
% Make \paragraph and \subparagraph free-standing
\ifx\paragraph\undefined\else
  \let\oldparagraph\paragraph
  \renewcommand{\paragraph}[1]{\oldparagraph{#1}\mbox{}}
\fi
\ifx\subparagraph\undefined\else
  \let\oldsubparagraph\subparagraph
  \renewcommand{\subparagraph}[1]{\oldsubparagraph{#1}\mbox{}}
\fi


\providecommand{\tightlist}{%
  \setlength{\itemsep}{0pt}\setlength{\parskip}{0pt}}\usepackage{longtable,booktabs,array}
\usepackage{calc} % for calculating minipage widths
% Correct order of tables after \paragraph or \subparagraph
\usepackage{etoolbox}
\makeatletter
\patchcmd\longtable{\par}{\if@noskipsec\mbox{}\fi\par}{}{}
\makeatother
% Allow footnotes in longtable head/foot
\IfFileExists{footnotehyper.sty}{\usepackage{footnotehyper}}{\usepackage{footnote}}
\makesavenoteenv{longtable}
\usepackage{graphicx}
\makeatletter
\def\maxwidth{\ifdim\Gin@nat@width>\linewidth\linewidth\else\Gin@nat@width\fi}
\def\maxheight{\ifdim\Gin@nat@height>\textheight\textheight\else\Gin@nat@height\fi}
\makeatother
% Scale images if necessary, so that they will not overflow the page
% margins by default, and it is still possible to overwrite the defaults
% using explicit options in \includegraphics[width, height, ...]{}
\setkeys{Gin}{width=\maxwidth,height=\maxheight,keepaspectratio}
% Set default figure placement to htbp
\makeatletter
\def\fps@figure{htbp}
\makeatother

\KOMAoption{captions}{tableheading}
\usepackage{wrapfig}
\usepackage{subcaption}
\usepackage{amsmath}
\usepackage{cancel}
\usepackage{hyperref}
\usepackage{tikz}
\usepackage{tabularx}
\renewcommand{\maketitle}{}
\makeatletter
\@ifpackageloaded{caption}{}{\usepackage{caption}}
\AtBeginDocument{%
\ifdefined\contentsname
  \renewcommand*\contentsname{Table of contents}
\else
  \newcommand\contentsname{Table of contents}
\fi
\ifdefined\listfigurename
  \renewcommand*\listfigurename{Figurliste}
\else
  \newcommand\listfigurename{Figurliste}
\fi
\ifdefined\listtablename
  \renewcommand*\listtablename{Tabelliste}
\else
  \newcommand\listtablename{Tabelliste}
\fi
\ifdefined\figurename
  \renewcommand*\figurename{Figur}
\else
  \newcommand\figurename{Figur}
\fi
\ifdefined\tablename
  \renewcommand*\tablename{Tabell}
\else
  \newcommand\tablename{Tabell}
\fi
}
\@ifpackageloaded{float}{}{\usepackage{float}}
\floatstyle{ruled}
\@ifundefined{c@chapter}{\newfloat{codelisting}{h}{lop}}{\newfloat{codelisting}{h}{lop}[chapter]}
\floatname{codelisting}{Listing}
\newcommand*\listoflistings{\listof{codelisting}{List of Listings}}
\makeatother
\makeatletter
\makeatother
\makeatletter
\@ifpackageloaded{caption}{}{\usepackage{caption}}
\@ifpackageloaded{subcaption}{}{\usepackage{subcaption}}
\makeatother
\ifLuaTeX
  \usepackage{selnolig}  % disable illegal ligatures
\fi
\usepackage{bookmark}

\IfFileExists{xurl.sty}{\usepackage{xurl}}{} % add URL line breaks if available
\urlstyle{same} % disable monospaced font for URLs
\hypersetup{
  colorlinks=true,
  linkcolor={blue},
  filecolor={Maroon},
  citecolor={Blue},
  urlcolor={Blue},
  pdfcreator={LaTeX via pandoc}}

\author{}
\date{}

\begin{document}


\newgeometry{left=0cm, right=0cm, top=0cm, bottom=0cm}
\vspace*{0.5cm} 
\hspace*{1.5cm}\includegraphics[width=10cm]{dokumentobjekter/texstuff/UiT_Logo_Bok_Bla_RGB.png} 


\begin{flushleft}
    \vspace*{0.5cm}
    \hspace*{2.5cm}{\color{black}\fontsize{11}{13.2}\selectfont Handelshøgskolen ved UiT \\[0.2em]
    \hspace*{2.5cm}\color{black}\fontsize{8}{13.2}\selectfont Fakultet for biovitenskap, fiskeri og økonomi \\[0.2em]
    \hspace*{2.5cm}\large{\color{black}\textbf{Mappeoppgave 2}}  \\[0.5em]
    \hspace*{2.5cm}\color{black}\fontsize{12}{14.4}\selectfont Mappeoppgave 2 i: SOK-2030 Næringsøkonomi og konkurransestrategi \\[0.5em]
\hspace*{2.5cm}\color{black}\fontsize{11}{13.2}\selectfont Kandidatnummer: 18 \\[0.5em]
    \hspace*{2.5cm}\color{black}\fontsize{11}{13.2}\selectfont Sok-2030, Vår 2024 \\[0.5em]
    \hspace*{2.0cm}
    \par}
\end{flushleft} 



\begin{tikzpicture}[remember picture, overlay]
    \node[anchor=south west, inner sep=0] at (current page.south west) {\includegraphics[width=\paperwidth]{dokumentobjekter/texstuff/forside_bilde.png}};
\end{tikzpicture}


\newgeometry{left=20mm, right=20mm, top=20mm, bottom=20mm}





\thispagestyle{plain}
\begin{center}
    \Large
    \textbf{Sammendrag}
\end{center}



\newpage
\hypersetup{linkcolor=black}
\renewcommand{\contentsname}{Innholdsfortegnelse}
\renewcommand*{\figureautorefname}{Figur}
\renewcommand*{\tableautorefname}{Tabell}
\tableofcontents
\newpage
\listoffigures
\listoftables
\hypersetup{linkcolor=blue}
\newpage

\section{1. Oppgave 1 (30\%)}\label{oppgave-1-30}

Olivita AS ble etablert i 2002 av to professorer fra Universitetet i
Tromsø (UiT). Selskapet tilbyr kosttilskuddet Olivita, som inneholder
omega-3 og er utviklet for å støtte hjerte, ledd og immunforsvar.
Produktet har vært patentbeskyttet frem til 2023, og Olivita har hatt
eksklusiv rett til produksjon av dette omega-3 produktet. Etter
patentets utløp har det nye selskapet Dr Choice AS kommet på markedet og
tilbyr Easy Choice Omega-3. I markedet for omega-3 produkter vil Olivita
AS fortsette å være en ledende aktør, mens Dr Choice AS vil utfordre som
en nykommer.

I dette marked er det følgende invers etterspørsel \[
P = 990 - \frac{1}{60}(q_O + q_C)
\] hvor \(q_O\) er antall solgte flasker med Olivita, \(q_C\) er antall
solgte flasker Easy Choice Omega-3 og \(P\) er pris per flaske av
Omega-3 produktene. I produksjon av Omega-3 produktene vil begge
bedriftene ha konstante marginalkostnader på kr 50 per produsert flaske.
Faste kostnader for begge bedriftene er på 3 millioner kroner.

\subsection{a) Hva blir optimal tilpasning i dette markedet når Olivita kan gjøre sine strategiske valg før konkurrenten, Dr Choice AS, gjør sitt valg?}\label{a-hva-blir-optimal-tilpasning-i-dette-markedet-nuxe5r-olivita-kan-gjuxf8re-sine-strategiske-valg-fuxf8r}

Da det er sekvensielle valg og de har samme kostnader så bruker jeg
stackelberg modellen som modellerer hvordan aktørene tar valg i en
sekvensiell rekkefølge. Stackelberg modellen er en modell som brukes for
å analysere hvilket kvantum som produseres av hver aktør i et marked når
en aktør tar sitt valg først og den andre aktøren tar sitt valg etterpå.

Vi har da profittfunksjonene for begge bedriftene ved å ta markedspris
per enhet minus marginalkostnad, multiplisert med kvantum solgt, minus
faste kostnader. \[
\pi_O=(P-50) \cdot q_O-3000000 \\
\pi_C=(P-50) \cdot q_C-3000000
\]

Og setter inn for \(P\) fra den inverse etterspørselsfunksjonen:

$\displaystyle π_{O} = q_{C} \left(- 0.0166666666666667 q_{C} - 0.0166666666666667 q_{O} + 940\right) - 3000000$

$\displaystyle π_{C} = q_{O} \left(- 0.0166666666666667 q_{C} - 0.0166666666666667 q_{O} + 940\right) - 3000000$

I stackelberg modellen så starter man med trinn 2, som er å finne
optimalt kvantum for Dr Choice AS.

Så jeg starter med å derivere profitten til Dr Choice AS med hensyn på
\(q_C\) og setter lik 0 for å finne optimalt kvantum for Dr Choice AS.

$\displaystyle - 0.0166666666666667 q_{C} - 0.0333333333333333 q_{O} + 940$

For å nå finne reaksjonsfunksjonen til Dr Choice AS så løser vi
ligningen for \(q_C\).

$\displaystyle 28200.0 - 0.5 q_{C}$

Vi setter så inn reaksjonsfunksjonen til Dr Choice AS i profitten til
Olivita.

$\displaystyle π_{O} = q_{C} \left(470.0 - 0.00833333333333333 q_{C}\right) - 3000000$

Vi finner så optimalt kvantum for Olivita ved å derivere profitten til
Olivita med hensyn på \(q_C\) og setter lik 0.

$\displaystyle 470.0 - 0.0166666666666667 q_{C}$

Da kan vi finne kvantum for Olivita ved å løse ligningen for \(q_C\).

$\displaystyle 28200.0$

reaksjon\_olivitas

Så setter vi inn kvantum for Olivita i

$\displaystyle 14100.0$

Vi kan nå finne markedsprisen ved å sette inn kvantum for begge
bedriftene i den inverse etterspørselen.

$\displaystyle 285.0$

Og profitten til begge bedriftene

$\displaystyle π_{O} = 3627000.0$

$\displaystyle π_{C} = 313500.0$

\subsection{b) Vil det være en fordel for Olivita å ha mulighet til å
gjøre sine valg før konkurrenten gjør sitt
valg?}\label{b-vil-det-vuxe6re-en-fordel-for-olivita-uxe5-ha-mulighet-til-uxe5-gjuxf8re-sine-valg-fuxf8r-konkurrenten-gjuxf8r-sitt-valg}

Det virker sånn men for å finne ut dette så vil jeg også teste dersom
Olivitas ikke får gjøre sitt valg først.

her masse stuff f f f f

\clearpage

\section{Oppgave 2 (70\%)}\label{oppgave-2-70}

Markedet for produksjon av mikroøl består av tre lokale bryggerier:
Graff Brygghus, Bryggeri 13 og Mack Mikrobryggeri. Etterspørselen i
dette markedet er gitt ved: \[
P = 175-4Q
\] hvor \(P\) er markedspris per flaske mikroøl, \(Q\) er totalt kvantum
(antall tusen flasker), som er summen av produksjonen til de tre
bryggeriene: \(Q = q_G + q_B + q_M\), der \(q_G\) er produsert kvantum
for Graff Brygghus, \(q_B\) er produsert kvantum for Bryggeri 13 og
\(q_M\) er produsert kvantum for Mack Mikrobryggeri.

Mack Mikrobryggeri, som er en del av Mack Ølbryggeri, har en mer
effektiv produksjonslinje enn de to andre, med konstante
marginalkostnader på 7 kr per flaske, mens Graff Brygghus og Bryggeri 13
har marginalkostnader på 10 kr per flaske. Alle tre mikrobryggeriene har
faste årlige kostnader på 300 000 kr. Styrene i selskapene Mack
Mikrobryggeri og Bryggeri 13 har startet samtaler knyttet til mulig
fusjon av disse to selskapene. Ved en fusjon vil all produksjon flyttes
til Mack Mikrobryggeri. De faste kostnadene vil også reduseres ved
sammenslåing av selskapene, og totalt utgjøre kr 500 000 per år for det
fusjonerte selskapet.

\subsection{a) Vil en slik fusjon være lønnsom for de fusjonerte
partene?}\label{a-vil-en-slik-fusjon-vuxe6re-luxf8nnsom-for-de-fusjonerte-partene}

Videre i oppgaven skal vi anta at fusjon mellom Mack Mikrobryggeri og
Bryggeri 13 blir gjennomført, og det nye selskapet vil operere under
navnet Mack Mikrobrygg 13. Markedet for produksjon av mikroøl vil da
bestå av to lokale produsenter: Mack Mikrobrygg 13 og Graff Brygghus.
For å styrke sin posisjon i markedet, investerer Graff Brygghus i nytt
og mer effektivt produksjonsutstyr, noe som reduserer deres variable
kostnader til kr 7 per flaske. Denne investeringen vil gi selskapet økte
faste kostnader på kr 200.000. Totale faste kostnader for begge
bryggeriene er da på kr 500.000 for hvert av selskapene.

I restaurantbransjen i Tromsø er Restaurant Gruppen Holdig (RGH) en
sentral aktør, som har monopol i sitt segment. RGH kjøper sitt mikroøl
fra de to lokale produsentene Mack Mikrobrygg 13 og Graff Brygghus. For
å drifte sine restauranter har RGH faste kostnader på kr 600.000.

Etterspørselen etter mikroøl i restaurantbransjen er lik: \[
P = 175 - 2Q
\] hvor \(Q\) er antall solgte flasker mikroøl (antall tusen flasker)
for RGH og \(P\) er prisen for en flaske mikroøl til sluttbruker. For å
ytterligere styrke sin posisjon i oppstrømsmarkedet, vurderer ledelsen i
Mack Mikrobrygg 13 en fusjon med konkurrenten Graff Brygghus. Det antas
at denne fusjonen ikke vil resultere i kostnadsbesparelser for
bryggeriene. Som konsulent for styret i Mack Mikrobrygg 13, er du bedt
om å analysere markedskonsekvensene av en potensiell fusjon mellom Mack
Mikrobrygg 13 og Graff Brygghus. Analysen skal omfatte en vurdering av
dagens markedstilpasning og en sammenligning med tilpasningen etter en
eventuell fusjon i oppstrømsmarkedet.

\subsection{b) Basert på din analyse, vil du anbefale styret i Mack
Mikrobrygg 13 å gjennomføre fusjon med Graff
Brygghus?}\label{b-basert-puxe5-din-analyse-vil-du-anbefale-styret-i-mack-mikrobrygg-13-uxe5-gjennomfuxf8re-fusjon-med-graff-brygghus}

\subsection{c) Hva blir de samfunnsøkonomiske konsekvensene av en fusjon
mellom Mack Mikrobrygg 13 og Graff
Brygghus?}\label{c-hva-blir-de-samfunnsuxf8konomiske-konsekvensene-av-en-fusjon-mellom-mack-mikrobrygg-13-og-graff-brygghus}

\clearpage

\section{Referanser}\label{referanser}

\appendix

\section {Appendix Generell KI bruk}

I løpet av koden så kan det ses mange \# kommentarer der det er skrevet
for eks ``\#fillbetween q1 and q2''. Når vi skriver kode i Visual Studio
Code så har vi en plugin som heter Github Copilot. Når vi skriver slike
kommentarer så kan den foresøke å fullføre kodelinjene mens vi skriver
de. Noen ganger klarer den det, men andre ikke. Det er vanskelig å
dokumentere hvert bruk der den er brukt siden det ``går veldig fort''
men siden vi ikke har fått på plass en slik dokumentasjon så kan all
python kode der det er brukt kommentarer antas som at det er brukt
Github Copilot. Nærmere info om dette KI verktøyet kan ses på
\url{https://github.com/features/copilot}

\clearpage

\section {Appendix Python kode Oppgave 1}

\clearpage

\section {Appendix Python kode Oppgave 2 a}

\clearpage

\section {Appendix Python kode Oppgave 2 b}

\clearpage

\section {Appendix Python kode Oppgave 2 c}



\end{document}
