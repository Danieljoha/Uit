% Options for packages loaded elsewhere
\PassOptionsToPackage{unicode}{hyperref}
\PassOptionsToPackage{hyphens}{url}
\PassOptionsToPackage{dvipsnames,svgnames,x11names}{xcolor}
%
\documentclass[
  12pt,
  a4paper,
  DIV=11,
  numbers=noendperiod]{scrartcl}

\usepackage{amsmath,amssymb}
\usepackage{iftex}
\ifPDFTeX
  \usepackage[T1]{fontenc}
  \usepackage[utf8]{inputenc}
  \usepackage{textcomp} % provide euro and other symbols
\else % if luatex or xetex
  \usepackage{unicode-math}
  \defaultfontfeatures{Scale=MatchLowercase}
  \defaultfontfeatures[\rmfamily]{Ligatures=TeX,Scale=1}
\fi
\usepackage{lmodern}
\ifPDFTeX\else  
    % xetex/luatex font selection
\fi
% Use upquote if available, for straight quotes in verbatim environments
\IfFileExists{upquote.sty}{\usepackage{upquote}}{}
\IfFileExists{microtype.sty}{% use microtype if available
  \usepackage[]{microtype}
  \UseMicrotypeSet[protrusion]{basicmath} % disable protrusion for tt fonts
}{}
\makeatletter
\@ifundefined{KOMAClassName}{% if non-KOMA class
  \IfFileExists{parskip.sty}{%
    \usepackage{parskip}
  }{% else
    \setlength{\parindent}{0pt}
    \setlength{\parskip}{6pt plus 2pt minus 1pt}}
}{% if KOMA class
  \KOMAoptions{parskip=half}}
\makeatother
\usepackage{xcolor}
\usepackage[top=20mm,left=20mm,heightrounded]{geometry}
\setlength{\emergencystretch}{3em} % prevent overfull lines
\setcounter{secnumdepth}{-\maxdimen} % remove section numbering
% Make \paragraph and \subparagraph free-standing
\ifx\paragraph\undefined\else
  \let\oldparagraph\paragraph
  \renewcommand{\paragraph}[1]{\oldparagraph{#1}\mbox{}}
\fi
\ifx\subparagraph\undefined\else
  \let\oldsubparagraph\subparagraph
  \renewcommand{\subparagraph}[1]{\oldsubparagraph{#1}\mbox{}}
\fi

\usepackage{color}
\usepackage{fancyvrb}
\newcommand{\VerbBar}{|}
\newcommand{\VERB}{\Verb[commandchars=\\\{\}]}
\DefineVerbatimEnvironment{Highlighting}{Verbatim}{commandchars=\\\{\}}
% Add ',fontsize=\small' for more characters per line
\usepackage{framed}
\definecolor{shadecolor}{RGB}{241,243,245}
\newenvironment{Shaded}{\begin{snugshade}}{\end{snugshade}}
\newcommand{\AlertTok}[1]{\textcolor[rgb]{0.68,0.00,0.00}{#1}}
\newcommand{\AnnotationTok}[1]{\textcolor[rgb]{0.37,0.37,0.37}{#1}}
\newcommand{\AttributeTok}[1]{\textcolor[rgb]{0.40,0.45,0.13}{#1}}
\newcommand{\BaseNTok}[1]{\textcolor[rgb]{0.68,0.00,0.00}{#1}}
\newcommand{\BuiltInTok}[1]{\textcolor[rgb]{0.00,0.23,0.31}{#1}}
\newcommand{\CharTok}[1]{\textcolor[rgb]{0.13,0.47,0.30}{#1}}
\newcommand{\CommentTok}[1]{\textcolor[rgb]{0.37,0.37,0.37}{#1}}
\newcommand{\CommentVarTok}[1]{\textcolor[rgb]{0.37,0.37,0.37}{\textit{#1}}}
\newcommand{\ConstantTok}[1]{\textcolor[rgb]{0.56,0.35,0.01}{#1}}
\newcommand{\ControlFlowTok}[1]{\textcolor[rgb]{0.00,0.23,0.31}{#1}}
\newcommand{\DataTypeTok}[1]{\textcolor[rgb]{0.68,0.00,0.00}{#1}}
\newcommand{\DecValTok}[1]{\textcolor[rgb]{0.68,0.00,0.00}{#1}}
\newcommand{\DocumentationTok}[1]{\textcolor[rgb]{0.37,0.37,0.37}{\textit{#1}}}
\newcommand{\ErrorTok}[1]{\textcolor[rgb]{0.68,0.00,0.00}{#1}}
\newcommand{\ExtensionTok}[1]{\textcolor[rgb]{0.00,0.23,0.31}{#1}}
\newcommand{\FloatTok}[1]{\textcolor[rgb]{0.68,0.00,0.00}{#1}}
\newcommand{\FunctionTok}[1]{\textcolor[rgb]{0.28,0.35,0.67}{#1}}
\newcommand{\ImportTok}[1]{\textcolor[rgb]{0.00,0.46,0.62}{#1}}
\newcommand{\InformationTok}[1]{\textcolor[rgb]{0.37,0.37,0.37}{#1}}
\newcommand{\KeywordTok}[1]{\textcolor[rgb]{0.00,0.23,0.31}{#1}}
\newcommand{\NormalTok}[1]{\textcolor[rgb]{0.00,0.23,0.31}{#1}}
\newcommand{\OperatorTok}[1]{\textcolor[rgb]{0.37,0.37,0.37}{#1}}
\newcommand{\OtherTok}[1]{\textcolor[rgb]{0.00,0.23,0.31}{#1}}
\newcommand{\PreprocessorTok}[1]{\textcolor[rgb]{0.68,0.00,0.00}{#1}}
\newcommand{\RegionMarkerTok}[1]{\textcolor[rgb]{0.00,0.23,0.31}{#1}}
\newcommand{\SpecialCharTok}[1]{\textcolor[rgb]{0.37,0.37,0.37}{#1}}
\newcommand{\SpecialStringTok}[1]{\textcolor[rgb]{0.13,0.47,0.30}{#1}}
\newcommand{\StringTok}[1]{\textcolor[rgb]{0.13,0.47,0.30}{#1}}
\newcommand{\VariableTok}[1]{\textcolor[rgb]{0.07,0.07,0.07}{#1}}
\newcommand{\VerbatimStringTok}[1]{\textcolor[rgb]{0.13,0.47,0.30}{#1}}
\newcommand{\WarningTok}[1]{\textcolor[rgb]{0.37,0.37,0.37}{\textit{#1}}}

\providecommand{\tightlist}{%
  \setlength{\itemsep}{0pt}\setlength{\parskip}{0pt}}\usepackage{longtable,booktabs,array}
\usepackage{calc} % for calculating minipage widths
% Correct order of tables after \paragraph or \subparagraph
\usepackage{etoolbox}
\makeatletter
\patchcmd\longtable{\par}{\if@noskipsec\mbox{}\fi\par}{}{}
\makeatother
% Allow footnotes in longtable head/foot
\IfFileExists{footnotehyper.sty}{\usepackage{footnotehyper}}{\usepackage{footnote}}
\makesavenoteenv{longtable}
\usepackage{graphicx}
\makeatletter
\def\maxwidth{\ifdim\Gin@nat@width>\linewidth\linewidth\else\Gin@nat@width\fi}
\def\maxheight{\ifdim\Gin@nat@height>\textheight\textheight\else\Gin@nat@height\fi}
\makeatother
% Scale images if necessary, so that they will not overflow the page
% margins by default, and it is still possible to overwrite the defaults
% using explicit options in \includegraphics[width, height, ...]{}
\setkeys{Gin}{width=\maxwidth,height=\maxheight,keepaspectratio}
% Set default figure placement to htbp
\makeatletter
\def\fps@figure{htbp}
\makeatother
% definitions for citeproc citations
\NewDocumentCommand\citeproctext{}{}
\NewDocumentCommand\citeproc{mm}{%
  \begingroup\def\citeproctext{#2}\cite{#1}\endgroup}
\makeatletter
 % allow citations to break across lines
 \let\@cite@ofmt\@firstofone
 % avoid brackets around text for \cite:
 \def\@biblabel#1{}
 \def\@cite#1#2{{#1\if@tempswa , #2\fi}}
\makeatother
\newlength{\cslhangindent}
\setlength{\cslhangindent}{1.5em}
\newlength{\csllabelwidth}
\setlength{\csllabelwidth}{3em}
\newenvironment{CSLReferences}[2] % #1 hanging-indent, #2 entry-spacing
 {\begin{list}{}{%
  \setlength{\itemindent}{0pt}
  \setlength{\leftmargin}{0pt}
  \setlength{\parsep}{0pt}
  % turn on hanging indent if param 1 is 1
  \ifodd #1
   \setlength{\leftmargin}{\cslhangindent}
   \setlength{\itemindent}{-1\cslhangindent}
  \fi
  % set entry spacing
  \setlength{\itemsep}{#2\baselineskip}}}
 {\end{list}}
\usepackage{calc}
\newcommand{\CSLBlock}[1]{\hfill\break\parbox[t]{\linewidth}{\strut\ignorespaces#1\strut}}
\newcommand{\CSLLeftMargin}[1]{\parbox[t]{\csllabelwidth}{\strut#1\strut}}
\newcommand{\CSLRightInline}[1]{\parbox[t]{\linewidth - \csllabelwidth}{\strut#1\strut}}
\newcommand{\CSLIndent}[1]{\hspace{\cslhangindent}#1}

\KOMAoption{captions}{tableheading}
\makeatletter
\@ifpackageloaded{caption}{}{\usepackage{caption}}
\AtBeginDocument{%
\ifdefined\contentsname
  \renewcommand*\contentsname{Table of contents}
\else
  \newcommand\contentsname{Table of contents}
\fi
\ifdefined\listfigurename
  \renewcommand*\listfigurename{List of Figures}
\else
  \newcommand\listfigurename{List of Figures}
\fi
\ifdefined\listtablename
  \renewcommand*\listtablename{List of Tables}
\else
  \newcommand\listtablename{List of Tables}
\fi
\ifdefined\figurename
  \renewcommand*\figurename{Figure}
\else
  \newcommand\figurename{Figure}
\fi
\ifdefined\tablename
  \renewcommand*\tablename{Table}
\else
  \newcommand\tablename{Table}
\fi
}
\@ifpackageloaded{float}{}{\usepackage{float}}
\floatstyle{ruled}
\@ifundefined{c@chapter}{\newfloat{codelisting}{h}{lop}}{\newfloat{codelisting}{h}{lop}[chapter]}
\floatname{codelisting}{Listing}
\newcommand*\listoflistings{\listof{codelisting}{List of Listings}}
\makeatother
\makeatletter
\makeatother
\makeatletter
\@ifpackageloaded{caption}{}{\usepackage{caption}}
\@ifpackageloaded{subcaption}{}{\usepackage{subcaption}}
\makeatother
\ifLuaTeX
  \usepackage{selnolig}  % disable illegal ligatures
\fi
\usepackage{bookmark}

\IfFileExists{xurl.sty}{\usepackage{xurl}}{} % add URL line breaks if available
\urlstyle{same} % disable monospaced font for URLs
\hypersetup{
  pdftitle={OBLIGATORISK INNLEVERING 1, STA-1001 2024},
  colorlinks=true,
  linkcolor={blue},
  filecolor={Maroon},
  citecolor={Blue},
  urlcolor={Blue},
  pdfcreator={LaTeX via pandoc}}

\title{OBLIGATORISK INNLEVERING 1, STA-1001 2024}
\author{}
\date{12-02-2024}

\begin{document}
\maketitle
\begin{abstract}
\hfill\break
\hfill\break
\hfill\break
\hfill\break
\hfill\break
\hfill\break
\hfill\break
\hfill\break
\hfill\break
\hfill\break
\hfill\break
\hfill\break
\hfill\break
\hfill\break
\hfill\break
\hfill\break
\hfill\break
\hfill\break
\hfill\break
\hfill\break
\hfill\break
\hfill\break
\hfill\break
\hfill\break
\hfill\break
\hfill\break
\hfill\break
\hfill\break
\hfill\break
\hfill\break
\hfill\break
\end{abstract}

\renewcommand*\contentsname{Innholdsfortegnelse}
{
\hypersetup{linkcolor=black}
\setcounter{tocdepth}{3}
\tableofcontents
}
\clearpage

\subsection{Oppgave 1:}\label{oppgave-1}

Gjør oppgave 1.18 i boka ved bruk av R. (Om du vil slippe å skrive inn
dataene ligger det ei fil karakterer.txt på Canvas.)

Bruk også R til å lage punktdiagram (funksjonen stripchart) og box-plott
for dataene.

\begin{Shaded}
\begin{Highlighting}[]
\FunctionTok{library}\NormalTok{(tidyverse)}
\FunctionTok{library}\NormalTok{(stringr)}
\NormalTok{karakterer }\OtherTok{\textless{}{-}} \StringTok{"23 60 79 32 57 74 52 70 82 36 80 77 81 95 41 65 92 85 55 76 52 10 64 75 78 25 80 98 81 67 41 71 83 54 64 72 88 62 74 43 60 78 89 76 84 48 84 90 15 79 34 67 17 82 69 74 63 80 85 61"}
\NormalTok{karakterer }\OtherTok{\textless{}{-}} \FunctionTok{str\_replace\_all}\NormalTok{(karakterer, }\StringTok{" "}\NormalTok{, }\StringTok{"}\SpecialCharTok{\textbackslash{}n}\StringTok{"}\NormalTok{)}


\NormalTok{karakterer }\OtherTok{\textless{}{-}} \FunctionTok{read.csv}\NormalTok{(}\AttributeTok{text=}\NormalTok{karakterer, }\AttributeTok{header=}\ConstantTok{FALSE}\NormalTok{) }

\NormalTok{karakterer }\OtherTok{\textless{}{-}}\NormalTok{ karakterer }\SpecialCharTok{\%\textgreater{}\%} 
  \FunctionTok{rename}\NormalTok{(}\AttributeTok{karakterer =}\NormalTok{ V1) }\SpecialCharTok{\%\textgreater{}\%} 
  \FunctionTok{mutate}\NormalTok{(}\AttributeTok{index =} \FunctionTok{row\_number}\NormalTok{())}


\NormalTok{karakterer }\SpecialCharTok{\%\textgreater{}\%} 
  \FunctionTok{ggplot}\NormalTok{(}\FunctionTok{aes}\NormalTok{(}\AttributeTok{x=}\NormalTok{karakterer, }\AttributeTok{y=}\NormalTok{index)) }\SpecialCharTok{+} 
  \FunctionTok{geom\_point}\NormalTok{()}\SpecialCharTok{+}
  \FunctionTok{labs}\NormalTok{(}\AttributeTok{y=}\StringTok{""}\NormalTok{, }\AttributeTok{title=}\StringTok{"Oppgave 1, punkt diagram"}\NormalTok{)}
\end{Highlighting}
\end{Shaded}

\includegraphics{STA1001_oblig1_V24_files/figure-pdf/unnamed-chunk-1-1.pdf}

\begin{Shaded}
\begin{Highlighting}[]
\NormalTok{karakterer }\SpecialCharTok{\%\textgreater{}\%} 
  \FunctionTok{ggplot}\NormalTok{(}\FunctionTok{aes}\NormalTok{(}\AttributeTok{x=}\NormalTok{karakterer)) }\SpecialCharTok{+}
  \FunctionTok{geom\_boxplot}\NormalTok{() }\SpecialCharTok{+}
  \FunctionTok{labs}\NormalTok{(}\AttributeTok{title=}\StringTok{"Oppgave 1, boxplot"}\NormalTok{, }\AttributeTok{x=}\StringTok{"Karakterer"}\NormalTok{, }\AttributeTok{y=}\StringTok{""}\NormalTok{)}
\end{Highlighting}
\end{Shaded}

\includegraphics{STA1001_oblig1_V24_files/figure-pdf/unnamed-chunk-2-1.pdf}

\subsection{Oppgave 2:}\label{oppgave-2}

I boken ``Sannsynlighetsregning og statistikk for høyere utdanning'' på
side 29 så har jeg benyttet meg av bevis 1.10 for hjelp av
utregning.\footnote{Kristensen \& Wikan (2019)}

\[
\sum_{i=1}^{n} (x_i - \bar{x})^2 = \sum_{i=1}^{n} x_i^2 - n\bar{x}^2 \tag{1}
\]

\[s^2 = \frac{\sum_{i=1}^{n} x_i^2 - n \cdot \bar x^2}{n-1}\]

For å vise dette, starter jeg med å gange ut venstre siden av ligning 1.

\[\sum_{i=1}^{n} (x_i - \bar x)^2\]

Denne deler vi så opp:

\[\sum_{i=1}^{n} x_i^2 - 2\bar x\sum_{i=1}^{n} x_i + \sum_{i=1}^{n} \bar x^2\]

Gitt at et aritmetisk gjennomsnitt er regnet med

\[\bar{x} = \frac{1}{n}\sum_{i=1}^{n} x_i \tag{2}\]

så kan jeg forenkle det andre leddet ved å bruke ligning 2.

\[- 2\bar x \sum_{i=1}^{n} x_i = -2 \cdot \frac{1}{n}\sum_{i=1}^{n} x_i \cdot \sum_{i=1}^{n} x_i\]
som da endelig gir meg

\[-2 \cdot \left(\frac{\sum_{i=1}^{n} x_i}{n}\right)^2 = -2 \cdot n \cdot \bar x^2\]

Setter den så inn igjen i ligning 1

\[\sum_{i=1}^{n} x_i^2 -2 \cdot n \cdot \bar x^2 + \sum_{i=1}^{n} \bar x^2\]

Så starter jeg på det tredje leddet og gitt at \(\bar x\) er konstant så
skriver jeg den om til

\[\sum_{i=1}^{n} \bar x^2 = n \cdot \bar x^2\]

Til slutt får jeg som endelig forenkles ut i ligning 3.
\[\sum_{i=1}^{n} x_i^2 -2 \cdot n \cdot \bar x^2 + n \cdot \bar x^2\]

\[= \sum_{i=1}^{n} x_i^2 - n\bar x^2 \tag{3}\]

\subsection{Oppgave 3:}\label{oppgave-3}

En artist skal gi ut et album. Problemet er at artisten har ti låter, og
må velge bort to av dem. Han vet heller ikke hvilken rekkefølge han vil
bruke. La oss kalle låtene: A B C D E F G H I J

\subsubsection{a) Om han skal velge 8 av de 10
låtene:}\label{a-om-han-skal-velge-8-av-de-10-luxe5tene}

\begin{itemize}
\tightlist
\item
  Hvor mange mulige kombinasjoner av 8 låter finnes det?
\end{itemize}

\begin{Shaded}
\begin{Highlighting}[]
\CommentTok{\#factorial(10)/(factorial(8)*factorial(10{-}8))}
\FunctionTok{choose}\NormalTok{(}\DecValTok{10}\NormalTok{, }\DecValTok{8}\NormalTok{)}
\end{Highlighting}
\end{Shaded}

\begin{verbatim}
[1] 45
\end{verbatim}

\begin{itemize}
\tightlist
\item
  Hvor mange mulige kombinasjoner finnes det om A, B og C skal være med?
\end{itemize}

\begin{Shaded}
\begin{Highlighting}[]
\FunctionTok{choose}\NormalTok{(}\DecValTok{7}\NormalTok{, }\DecValTok{5}\NormalTok{)}
\end{Highlighting}
\end{Shaded}

\begin{verbatim}
[1] 21
\end{verbatim}

\begin{Shaded}
\begin{Highlighting}[]
\CommentTok{\#factorial(7)/(factorial(5)*factorial(7{-}5))}
\end{Highlighting}
\end{Shaded}

\subsubsection{b) Om han skal velge 8 av de 10
låtene:}\label{b-om-han-skal-velge-8-av-de-10-luxe5tene}

\begin{itemize}
\tightlist
\item
  Hvor mange mulige rekkefølger av 8 låter finns det?
\end{itemize}

\begin{Shaded}
\begin{Highlighting}[]
\CommentTok{\#(factorial(10)/(factorial(8)*factorial(10{-}8)))*factorial(8)}
\FunctionTok{choose}\NormalTok{(}\DecValTok{10}\NormalTok{, }\DecValTok{8}\NormalTok{)}\SpecialCharTok{*}\FunctionTok{factorial}\NormalTok{(}\DecValTok{8}\NormalTok{)}
\end{Highlighting}
\end{Shaded}

\begin{verbatim}
[1] 1814400
\end{verbatim}

\begin{itemize}
\tightlist
\item
  Hvor mange av rekkefølgene er i alfabetisk ordning?
\end{itemize}

\begin{Shaded}
\begin{Highlighting}[]
\CommentTok{\#factorial(10)/(factorial(8)*factorial(10{-}8))}
\FunctionTok{choose}\NormalTok{(}\DecValTok{10}\NormalTok{, }\DecValTok{8}\NormalTok{)}
\end{Highlighting}
\end{Shaded}

\begin{verbatim}
[1] 45
\end{verbatim}

Merk at i oppgave c) skal det bare velges ut 6 låter

\subsubsection{c) Om han skal velge 6 av de 10 låtene, men må ha med B
om han har med A (om motsatt), og
kan}\label{c-om-han-skal-velge-6-av-de-10-luxe5tene-men-muxe5-ha-med-b-om-han-har-med-a-om-motsatt-og-kan}

ikke ha med J om han har med I (og motsatt): - Hvor mange mulige
kombinasjoner av 6 låter finns det?

\begin{Shaded}
\begin{Highlighting}[]
\NormalTok{ab\_uten\_i\_og\_j }\OtherTok{\textless{}{-}} \FunctionTok{choose}\NormalTok{(}\DecValTok{6}\NormalTok{, }\DecValTok{4}\NormalTok{)}


\NormalTok{ab\_enten\_ij }\OtherTok{\textless{}{-}} \FunctionTok{choose}\NormalTok{(}\DecValTok{6}\NormalTok{, }\DecValTok{3}\NormalTok{) }\SpecialCharTok{*} \DecValTok{2} 

\NormalTok{uten\_abij }\OtherTok{\textless{}{-}} \FunctionTok{choose}\NormalTok{(}\DecValTok{6}\NormalTok{, }\DecValTok{6}\NormalTok{)}


\NormalTok{i\_eller\_j\_uten\_ab }\OtherTok{\textless{}{-}} \FunctionTok{choose}\NormalTok{(}\DecValTok{6}\NormalTok{, }\DecValTok{5}\NormalTok{) }\SpecialCharTok{*} \DecValTok{2}



\FunctionTok{print}\NormalTok{(ab\_uten\_i\_og\_j }\SpecialCharTok{+}\NormalTok{ ab\_enten\_ij }\SpecialCharTok{+}\NormalTok{ uten\_abij }\SpecialCharTok{+}\NormalTok{ i\_eller\_j\_uten\_ab)}
\end{Highlighting}
\end{Shaded}

\begin{verbatim}
[1] 68
\end{verbatim}

\begin{Shaded}
\begin{Highlighting}[]
\CommentTok{\#(choose(6, 4) + choose(6, 3) + choose(6, 3) + choose(6, 6) + choose(6, 5)+ choose(6, 5))}
\end{Highlighting}
\end{Shaded}

\subsubsection{d) Om han skal velge 8 av de 10 låtene og velger
tilfeldig:}\label{d-om-han-skal-velge-8-av-de-10-luxe5tene-og-velger-tilfeldig}

\begin{itemize}
\tightlist
\item
  Hva er sannsynligheten for at A kommer med?
\end{itemize}

\begin{Shaded}
\begin{Highlighting}[]
\FunctionTok{choose}\NormalTok{(}\DecValTok{9}\NormalTok{, }\DecValTok{7}\NormalTok{) }\SpecialCharTok{/} \FunctionTok{choose}\NormalTok{(}\DecValTok{10}\NormalTok{, }\DecValTok{8}\NormalTok{)}
\end{Highlighting}
\end{Shaded}

\begin{verbatim}
[1] 0.8
\end{verbatim}

\begin{itemize}
\tightlist
\item
  Hva er sannsynligheten for at minst ́en av A og B kommer med?
\end{itemize}

\begin{Shaded}
\begin{Highlighting}[]
\DecValTok{1} \SpecialCharTok{{-}}\NormalTok{ (}\FunctionTok{choose}\NormalTok{(}\DecValTok{8}\NormalTok{, }\DecValTok{8}\NormalTok{) }\SpecialCharTok{/} \FunctionTok{choose}\NormalTok{(}\DecValTok{10}\NormalTok{, }\DecValTok{8}\NormalTok{))}
\end{Highlighting}
\end{Shaded}

\begin{verbatim}
[1] 0.9777778
\end{verbatim}

\subsection{Oppgave 4}\label{oppgave-4}

Det er kjent at 10\% av laksepopulasjonen i et vassdrag har en bestemt
parasitt, her kalt parasitt A. Det er også kjent at 20\% av populasjonen
har en parasitt kalt parasitt B. Vi veit og at av de laksene som har
parasitt A (gitt parasitt A) vil 50\% ha parasitt B (betinga
sannsynlighet 0.5).

\subsubsection{a) Er hendelsene A = ``laks har parasitt A'' og B =
``laks har parasitt B''
uavhengige?}\label{a-er-hendelsene-a-laks-har-parasitt-a-og-b-laks-har-parasitt-b-uavhengige}

Dersom P(B\textbar A) = P(B) eller P(A\textbar B) = P(A) så er de
uavhengige. Men P(B) = 0.2 men P(B\textbar A) = 0.5 så de er ikke
uavhengige.

Er de disjunkte?

Nei, fordi P(A \(\cap\) B) \textgreater{} 0.

\subsubsection{\texorpdfstring{b) Regn ut sannsynligheten for at en laks
har begge parasittene P (A \(\cap\)
B).}{b) Regn ut sannsynligheten for at en laks har begge parasittene P (A \textbackslash cap B).}}\label{b-regn-ut-sannsynligheten-for-at-en-laks-har-begge-parasittene-p-a-cap-b.}

\begin{Shaded}
\begin{Highlighting}[]
\NormalTok{b\_gitt\_a }\OtherTok{=} \FloatTok{0.5}
\NormalTok{p\_a }\OtherTok{=} \FloatTok{0.1}
\NormalTok{p\_b }\OtherTok{=} \FloatTok{0.2}

\NormalTok{p\_og\_b }\OtherTok{=}\NormalTok{ b\_gitt\_a }\SpecialCharTok{*}\NormalTok{ p\_a}
\NormalTok{p\_og\_b}
\end{Highlighting}
\end{Shaded}

\begin{verbatim}
[1] 0.05
\end{verbatim}

Regn ut sannsynligheten for at en laks har minst en parasitt.

\begin{Shaded}
\begin{Highlighting}[]
\NormalTok{p\_eller\_b }\OtherTok{=}\NormalTok{ p\_a }\SpecialCharTok{+}\NormalTok{ p\_b }\SpecialCharTok{{-}}\NormalTok{ p\_og\_b}
\NormalTok{p\_eller\_b}
\end{Highlighting}
\end{Shaded}

\begin{verbatim}
[1] 0.25
\end{verbatim}

\subsubsection{c) Gitt at en laks har minst en parasitt, hva er
sannsynligheten for at den har parasitt
A?}\label{c-gitt-at-en-laks-har-minst-en-parasitt-hva-er-sannsynligheten-for-at-den-har-parasitt-a}

\begin{Shaded}
\begin{Highlighting}[]
\NormalTok{p\_a\_gitt\_eller\_b }\OtherTok{=}\NormalTok{ p\_a }\SpecialCharTok{/}\NormalTok{ p\_eller\_b}
\NormalTok{p\_a\_gitt\_eller\_b}
\end{Highlighting}
\end{Shaded}

\begin{verbatim}
[1] 0.4
\end{verbatim}

Gitt at en laks ikke har parasitt A, hva er sannsynligheten for at den
har parasitt B?

\begin{Shaded}
\begin{Highlighting}[]
\NormalTok{p\_b\_gitt\_kompli\_a }\OtherTok{=}\NormalTok{ (p\_b }\SpecialCharTok{{-}}\NormalTok{ p\_og\_b) }\SpecialCharTok{/}\NormalTok{ (}\DecValTok{1} \SpecialCharTok{{-}}\NormalTok{ p\_a)}
\NormalTok{p\_b\_gitt\_kompli\_a}
\end{Highlighting}
\end{Shaded}

\begin{verbatim}
[1] 0.1666667
\end{verbatim}

Vi tenker oss at n = 10 laks blir sjekka for parasitt A.

\subsubsection{d) Hva er sannsynligheten for at minst ́en laks har
parasitt
A?}\label{d-hva-er-sannsynligheten-for-at-minst-en-laks-har-parasitt-a}

\begin{Shaded}
\begin{Highlighting}[]
\DecValTok{1} \SpecialCharTok{{-}}\NormalTok{ (}\DecValTok{1} \SpecialCharTok{{-}} \FloatTok{0.1}\NormalTok{)}\SpecialCharTok{\^{}}\DecValTok{10}
\end{Highlighting}
\end{Shaded}

\begin{verbatim}
[1] 0.6513216
\end{verbatim}

Hva er sannsynligheten for at akkurat ́en laks har parasitt A?

\begin{Shaded}
\begin{Highlighting}[]
\FunctionTok{choose}\NormalTok{(}\DecValTok{10}\NormalTok{, }\DecValTok{1}\NormalTok{) }\SpecialCharTok{*} \FloatTok{0.1}\SpecialCharTok{\^{}}\DecValTok{1} \SpecialCharTok{*}\NormalTok{ (}\DecValTok{1} \SpecialCharTok{{-}} \FloatTok{0.1}\NormalTok{)}\SpecialCharTok{\^{}}\NormalTok{(}\DecValTok{10{-}1}\NormalTok{) }\CommentTok{\#må dobbelsjekke når jeg har tid. eksemepl 5.3 s167}
\end{Highlighting}
\end{Shaded}

\begin{verbatim}
[1] 0.3874205
\end{verbatim}

\subsection{5}\label{section}

For en type backup-batterier er det kjent at levetida (tida, i år, det
tar før de svikter) er gitt ved sannsynlighetstettheten

\[
f(x) = \frac{2}{5} x e^{-\frac{1}{2}x^2}, \quad x > 0
\]

\subsubsection{a) Skisser tettheten i
R.}\label{a-skisser-tettheten-i-r.}

Vis at dette er ei ekte fordeling.

\begin{Shaded}
\begin{Highlighting}[]
\FunctionTok{curve}\NormalTok{((}\DecValTok{2}\SpecialCharTok{/}\DecValTok{5}\NormalTok{)}\SpecialCharTok{*}\NormalTok{x}\SpecialCharTok{*}\FunctionTok{exp}\NormalTok{(}\SpecialCharTok{{-}}\DecValTok{1}\SpecialCharTok{/}\DecValTok{5}\SpecialCharTok{*}\NormalTok{x}\SpecialCharTok{\^{}}\DecValTok{2}\NormalTok{), }\AttributeTok{from =} \DecValTok{0}\NormalTok{, }\AttributeTok{to =} \DecValTok{5}\NormalTok{)}
\end{Highlighting}
\end{Shaded}

\includegraphics{STA1001_oblig1_V24_files/figure-pdf/unnamed-chunk-16-1.pdf}

\begin{Shaded}
\begin{Highlighting}[]
\FunctionTok{integrate}\NormalTok{(}\ControlFlowTok{function}\NormalTok{(x) (}\DecValTok{2}\SpecialCharTok{/}\DecValTok{5}\NormalTok{)}\SpecialCharTok{*}\NormalTok{x}\SpecialCharTok{*}\FunctionTok{exp}\NormalTok{(}\SpecialCharTok{{-}}\DecValTok{1}\SpecialCharTok{/}\DecValTok{5}\SpecialCharTok{*}\NormalTok{x}\SpecialCharTok{\^{}}\DecValTok{2}\NormalTok{), }\DecValTok{0}\NormalTok{, }\ConstantTok{Inf}\NormalTok{)}
\end{Highlighting}
\end{Shaded}

\begin{verbatim}
1 with absolute error < 2.5e-07
\end{verbatim}

Areal under kurven er 1, så det er en ekte fordeling.

Finn den kumulative fordelingsfunksjonen, ( F(x) ), og skissér også
denne.

\begin{Shaded}
\begin{Highlighting}[]
\FunctionTok{curve}\NormalTok{(}\DecValTok{1} \SpecialCharTok{{-}} \FunctionTok{exp}\NormalTok{(}\SpecialCharTok{{-}}\DecValTok{1}\SpecialCharTok{/}\DecValTok{5}\SpecialCharTok{*}\NormalTok{x}\SpecialCharTok{\^{}}\DecValTok{2}\NormalTok{), }\AttributeTok{from =} \DecValTok{0}\NormalTok{, }\AttributeTok{to =} \DecValTok{5}\NormalTok{)}
\end{Highlighting}
\end{Shaded}

\includegraphics{STA1001_oblig1_V24_files/figure-pdf/unnamed-chunk-18-1.pdf}

\subsubsection{b) Når er det siste tidspunktet du fremdeles kan være
90\% sikker på at et batteri
virker?}\label{b-nuxe5r-er-det-siste-tidspunktet-du-fremdeles-kan-vuxe6re-90-sikker-puxe5-at-et-batteri-virker}

\begin{Shaded}
\begin{Highlighting}[]
\FunctionTok{uniroot}\NormalTok{(}\ControlFlowTok{function}\NormalTok{(x) }\DecValTok{1} \SpecialCharTok{{-}} \FunctionTok{exp}\NormalTok{(}\SpecialCharTok{{-}}\DecValTok{1}\SpecialCharTok{/}\DecValTok{5}\SpecialCharTok{*}\NormalTok{x}\SpecialCharTok{\^{}}\DecValTok{2}\NormalTok{) }\SpecialCharTok{{-}} \FloatTok{0.9}\NormalTok{, }\FunctionTok{c}\NormalTok{(}\DecValTok{0}\NormalTok{, }\DecValTok{5}\NormalTok{))}
\end{Highlighting}
\end{Shaded}

\begin{verbatim}
$root
[1] 3.393071

$f.root
[1] 1.158728e-07

$iter
[1] 9

$init.it
[1] NA

$estim.prec
[1] 6.103516e-05
\end{verbatim}

3.4år?

\subsubsection{c) Finn sannsynligheten for at et tilfeldig valgt batteri
skal virke etter ett
år.}\label{c-finn-sannsynligheten-for-at-et-tilfeldig-valgt-batteri-skal-virke-etter-ett-uxe5r.}

\begin{Shaded}
\begin{Highlighting}[]
\FunctionTok{exp}\NormalTok{(}\SpecialCharTok{{-}}\DecValTok{1}\SpecialCharTok{/}\DecValTok{5}\SpecialCharTok{*}\DecValTok{1}\SpecialCharTok{\^{}}\DecValTok{2}\NormalTok{)}
\end{Highlighting}
\end{Shaded}

\begin{verbatim}
[1] 0.8187308
\end{verbatim}

Om du inspiserer et batteri etter ett år og finner at det virker, hva er
sannsynligheten for at det ikke vil virke om du inspiserer det ett år
seinere?

\begin{Shaded}
\begin{Highlighting}[]
\DecValTok{1} \SpecialCharTok{{-}}\NormalTok{ (}\FunctionTok{exp}\NormalTok{(}\SpecialCharTok{{-}}\DecValTok{1}\SpecialCharTok{/}\DecValTok{5}\SpecialCharTok{*}\DecValTok{2}\SpecialCharTok{\^{}}\DecValTok{2}\NormalTok{) }\SpecialCharTok{/} \FunctionTok{exp}\NormalTok{(}\SpecialCharTok{{-}}\DecValTok{1}\SpecialCharTok{/}\DecValTok{5}\SpecialCharTok{*}\DecValTok{1}\SpecialCharTok{\^{}}\DecValTok{2}\NormalTok{))}
\end{Highlighting}
\end{Shaded}

\begin{verbatim}
[1] 0.4511884
\end{verbatim}

\subsection{6}\label{section-1}

Vi antar følgende simultane sannsynlighetstetthet for de stokastiske
variablene X og Y:

\[
f(x,y) = 
\begin{cases} 
2e^{-(2x+y)}, & x > 0, y > 0 \\
0, & \text{ellers}
\end{cases}
\]

\subsubsection{a) Finn de marginale fordelingene
(sannsynlighetstetthetene) for X og
Y?}\label{a-finn-de-marginale-fordelingene-sannsynlighetstetthetene-for-x-og-y}

\[ f_X(x) = \int_0^\infty 2e^{-(2x+y)} dy = 2e^{-2x} \int_0^\infty e^{-y} dy = 2e^{-2x} \]

Hva blir den betinga fordelinga (sannsynlighetstettheten) for X?

Er X og Y uavhengige?

Nå er vi interessert i hva som blir summen av X og Y.

\begin{Shaded}
\begin{Highlighting}[]
\CommentTok{\# integrate(function(x) 2*exp({-}2*x), 0, Inf)}
\CommentTok{\# integrate(function(y) 2*exp({-}y), 0, Inf)}
\end{Highlighting}
\end{Shaded}

\subsubsection{b) Finn ved integrasjon av den simultane
sannsynlighetstettheten}\label{b-finn-ved-integrasjon-av-den-simultane-sannsynlighetstettheten}

( f(x,y) ) hva som blir sannsynligheten for at summen av X og Y er under
2. Det vil si finn ( P(X + Y \textless{} 2) ).

\clearpage

\subsection{Referanser}\label{referanser}

\phantomsection\label{refs}
\begin{CSLReferences}{1}{0}
\bibitem[\citeproctext]{ref-kristensen_2019_sannsynlighetsregning}
Kristensen, Ø. \& Wikan, A. (2019). \emph{Sannsynlighetsregning og
statistikk - for høyere utdanning} (2nd ed., p. 29). Fagbokforlaget.

\end{CSLReferences}



\end{document}
