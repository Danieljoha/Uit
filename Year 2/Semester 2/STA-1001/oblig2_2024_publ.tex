\documentclass[10pt,a4paper]{article}
\usepackage[margin=25mm]{geometry}
\pagestyle{plain}

\begin{document}

\begin{center}
\section*{OBLIGATORISK INNLEVERING 2, STA-1001 2024
\\
INNLEVERINGSFRIST TIRSDAG 12/3 KL 23:59}
\end{center}
Oppgavene er fra kapittel 4 - 7.
{\O}vinga leveres som pdf i Canvas.
\\
\subsection*{1}
De stokastiske variablene $X$ og $Y$ har simultan sannsynlighetstetthet
\[
f(x,y) = \left\{\begin{array}{ll}
\frac{1}{2}\sqrt{y}\,e^{-x\,y}, & 0<x<\infty,\quad 1<y<4.
\\
0, & \mbox{ellers.}
\end{array}\right.
\]
\begin{itemize}
\item[a)\protect{\hspace{3.5mm}}]
Vis at fordelinga for $Y$ (marginalfordeling) har sannsynlighetstetthet
\[
h(y) = \frac{1}{2\sqrt{y}},\quad 1 < y < 4.
\]
\end{itemize}
\begin{itemize}
\item[b)\protect{\hspace{3.5mm}}]
Finn forventningsverdi, varians og standardavvik for $Y$.
\end{itemize}
\begin{itemize}
\item[c)\protect{\hspace{3.5mm}}]
Bruk den simultane sannsynlighetstettheten til {\aa} finne $E(XY)$ og $E(X)$.
\\
Hva blir kovariansen mellom $X$ og $Y$?
\end{itemize}
Vi er interessert i den stokastiske variabelen $1/Y$.
\begin{itemize}
\item[d)\protect{\hspace{3.5mm}}]
Finn forventningsverdi og varians for $1/Y$.
\end{itemize}



\subsection*{2}
Ola samler p{\aa} spillerkort av fotballspillere (fotballkort).
Korta kj{\o}per han ett og ett av gangen, men de er
innpakka s{\aa} han veit ikke p{\aa} forhand om det
er et kort han allerede har eller ikke.
\\
Vi g{\aa}r ut fra at alle korta han kj{\o}per er fra en spesiell serie med 10 ulike spillerkort,
at det i hvert kj{\o}p er lik sannsynlighet for alle de 10 ulike spillerkorta,
og at resultata av kortkj{\o}pa er uavhengige.
\\
Ola {\o}nsker seg framfor alt ett bestemt spillerkort (vi kaller han spiller A).
La $X$ v{\ae}re antall kort han m{\aa} kj{\o}pe til han f{\aa}r dette kortet.
\begin{itemize}
\item[a)\protect{\hspace{3.5mm}}] 
Forklar hvorfor antakelsene til geometrisk fordeling er oppfylt.
\\
Hva er sannsynligheten for at han f{\aa}r spillerkortet p{\aa} det fjerde kj{\o}pet?
\\
Dersom han ikke fikk spillerkortet p{\aa} det f{\o}rste kj{\o}pet,
hva er sannslynligheten for at han skal f{\aa} det p{\aa} det femte?
%Hva er sannsynligheten for at han ikke f{\aa}r spillerkortet p{\aa} de to f{\o}rste kj{\o}pa?
\end{itemize}
Vi vil se litt p{\aa} en teoretisk egenskap ved geometrisk fordeling.
Vi antar i b) at antall kj{\o}pte kort f{\o}lger
geometrisk fordeling med generell sannsynlighet $p$.
\begin{itemize}
\item[b)\protect{\hspace{3.5mm}}] 
Bruk punktsannsynligheten til geometrisk fordeling til {\aa} vise at
\[
E(X) = \frac{1}{p}
\]
\end{itemize}
La oss n{\aa} tenke oss at det er to spillerkort (spiller A og B) Ola spesielt {\o}nsker seg, 
og at han kj{\o}per kort til han har f{\aa}tt begge.
La $X$ v{\ae}re antall kj{\o}p til han f{\aa}r det f{\o}rste av disse spillerkorta,
og la $Y$ v{\ae}re antall kj{\o}p etter han fikk det f{\o}rste til han f{\aa}r det andre.
La $T$ v{\ae}re hvor mange kort han m{\aa} kj{\o}pe totalt for {\aa} f{\aa} disse to spillerkorta:
\[
T = X + Y
\]
\begin{itemize}
\item[c)\protect{\hspace{3.5mm}}] 
Argument\'er kort for hvorfor $X$ og $Y$ er uavhengige,
og hva som blir fordeling for hver enkelt.
\\
Bruk formler for forventning/varians i geometrisk fordeling til {\aa}
finne forventning og varians til $T$.
\end{itemize}
La oss n{\aa} tenke oss at Ola ikke kan kjøpe ett og ett kort, men pakker av n = 20
kort av gangen, forutsetningene er ellers de samme som f{\o}r. Han kjøper ei slik
pakke.
\begin{itemize}
\item[d)] 
Argument\'er for hva som er fordelinga for antallet kort i pakka som er med spiller A?
\\
Hva er sansynligheten for at Ola f{\aa}r minst ett kort med spiller A i pakka?
\item[e)]
Hva er forventa antall ulike spillerkort i pakka? (Hint: Indikatorvariabler
for hvert spillerkort.)
\end{itemize}



\subsection*{3}
Vi definerer en enhet av en r{\aa}vare (kan v{\ae}re gull, olje, korn, etc)
som den mengden du f{\aa}r kj{\o}pt for $M$ NOK (det er ikke viktig hva $M$ er).
Du regner med at om du kj{\o}per en enhet av vare A
blir fortjenesta etter ett {\aa}r normalfordelt
med forventning $\mu_A=50$ NOK og standardavvik $\sigma_A=100$ NOK.
Vi kaller fortjenesta $X$:
\begin{itemize}
\item[a)]
Om du kj{\o}per en slik enhet av vare A hva blir sannsynligheten for
\\
- at du ikke g{\aa}r med tap (det vil si f{\aa}r positiv fortjeneste)?
\\
- at du ikke g{\aa}r med tap og at fortjenesten er mindre enn 150? 
\\
- at du f{\aa}r fortjeneste mindre enn 150, 
gitt at du ikke g{\aa}r med tap?
\end{itemize}
\begin{itemize}
\item[b)]
Finn fortjenesten som er slik at du er 80\% sikker p{\aa} {\aa} f{\aa} en fortjeneste over dette. 
\end{itemize}
(I oppgave c) er fortsatt standardavviket 100 NOK.)
\begin{itemize}
\item[c)]
Hva m{\aa} forventningsverdien v{\ae}re for {\aa} ha
en sannsynlighet p{\aa} 90\% for ikke {\aa} g{\aa} med tap?
\end{itemize}



\subsection*{4}
Vi definerer igjen en enhet av en vare 
som den mengden du f{\aa}r kj{\o}pt for $M$ NOK.
N{\aa} vil du kj{\o}pe to varer, vare A og vare B,
og du regner med at fortjenestene for en enhet etter ett {\aa}r er respektive
\[
X \sim \mbox{N}(\mu_A,\sigma_A)
\qquad\mbox{og}\qquad
Y \sim \mbox{N}(\mu_B,\sigma_B)
\]
Vi antar \`og at $X$ og $Y$ er uavhengige
\\
Vi vil se p{\aa} total fortjeneste etter ett {\aa}r om du til sammen har
kj{\o}pt for bel{\o}pet $M$, men bruker 30\% p{\aa} vare A og 70\% 
p{\aa} vare B, alts{\aa} total fortjeneste gitt ved:
\[
T = 0.3\cdot X + 0.7\cdot Y
\]
I oppgave a) og b) g{\aa}r vi ut fra at
$\mu_A=50$, $\mu_B=50$, $\sigma_A=100$ og $\sigma_B=50$.
\begin{itemize}
\item[a)]
Finn forventningsverdien og variansen til den totale fortjenesta.
\\
Hva blir fordelinga til den totale fortjenesta?
\end{itemize}
\begin{itemize}
\item[b)]
Bruk fordelinga fra a) til {\aa} finne et 
95\% prediksjonsintervall for den totale fortjenesta.
\end{itemize}
N{\aa} g{\aa}r vi ut fra at du vil kj{\o}pe varer av type A og B 
slik at total forjeneste blir
\[
T = aX + bY,
\]
der $a+b=1$ og $a$ og $b$ er ikke-negative.
\\
Du {\o}nsker {\aa} finne hva $a$ (og $b$)  m{\aa} v{\ae}re for at
variansen i fortjenesta skal bli minst mulig.
\begin{itemize}
\item[c)]
Hva blir uttrykket for variansen til $T$?
\\
Hva er den verdien av $a$ (uttrykt ved $\sigma_A$ og $\sigma_B$)
som gj{\o}r variansen for fortjenesta minst mulig?
\\
Hva blir variansen til fortjenesta (uttrykt ved $\sigma_A$ og $\sigma_B$) med denne verdien for $a$?
\\
{\bf Frivillig}: Anta igjen $\sigma_A=100$, $\sigma_B=50$, og lag en figur for variansen til $T$ 
som funksjon av $a$.
\end{itemize}

















\end{document}

